\chapter[Conclusion]{Conclusion}
\label{ch:conclusion}

Stellar streams are a unique tool for measuring the structure of mass around galaxies where the density of visible tracers is low. The work presented in this \article\ has shown that it will soon be possible to infer the 3D shape and distribution of dark matter around the Milky Way by modeling the many streams observed in the Galactic halo. This goal will be made possible by near-future catalogs of 6D kinematic measurements for individual stars in these streams from surveys like the \gaia\ mission. We have also shown that dynamical chaos can dramatically alter the morphological evolution of streams. Since the amount and significance of chaos depends strongly on the Galactic gravitational potential, this supports developing a new method for constraining the potential based purely on the morphologies of the many long, thin streams presently seen around the Milky Way. In this section, we summarize and discuss the key results presented in this \article\ and discuss future directions for this work.

\section{Summary of Results} \label{sec:summary}

% Rewinder 1
In Chapter \ref{ch:rewinder1}, we developed a new algorithm for using precise kinematic measurements of stars in stellar streams to measure the parameters of the underlying host galaxy's gravitational potential. This work was motivated by the prospect of combining extremely precise distance measurements for RR Lyrae stars with proper motion measurements from the \gaia\ mission and ground-based radial velocities to obtain full-space kinematics for samples of stars in stellar streams in the Galactic halo. The algorithm operates with the simple assumption that stars observed in a stellar stream were once part of the same progenitor system; by integrating the orbits of the progenitor system and stream stars backwards in time, in more correct host galaxy potential models the stars should come closer (in 6D phase-space) to the progenitor. We demonstrated that by modeling observations of a simulation of the Sagittarius stream with realistic but optimistic uncertainties, the true potential parameters can be inferred with small uncertainties and negligible biases. While this is encouraging and demonstrates the power of using dynamically cold structures for dynamical inference, it does is not a proper likelihood function and therefore cannot be used when there are missing or poorly measured data dimensions.

% Rewinder 2
In Chapter \ref{ch:rewinder2}, we used the ideas presented in Chapter~\ref{ch:rewinder1} to develop a probabilistic model for stellar streams. The key enhancement of this new approach is that it is written as a likelihood function with priors on the parameters so that missing or poorly-measured data can be incorporated into the inference of the potential parameters. \rewinder\ is distinct from many other stream-modeling methods because it (1) makes no assumption about the underlying form or integrability of the host galaxy potential, (2) non-parametrically infers the mass-loss history of the progenitor system, and (3) models the distribution function of stars \emph{at the time of stripping} rather than at present-day and can therefore analytically evaluate the likelihood without numerically reconstructing the density field. We tested \rewinder\ with a simulation observed with different assumptions about the tracer population (different data qualities) and showed that in all cases, all input potential parameters are successfully recovered: mass and length scaling but also the shape parameters of the input triaxial halo. 

% Chaos / stream-morphology
In Chapter \ref{ch:chaos-morphology}, we showed that even when chaos is not important for restructuring the global orbit structure of a galaxy, chaos can greatly enhance the density evolution of stellar streams over just a few orbital times. This suggests that the morphology of tidal streams \emph{alone} can constrain the significance of chaos along the orbits of the progenitor systems, thereby placing constraints on the global properties of the gravitational potential. For example, different dark matter particle candidates (WIMPs vs. axions) predict vastly different amounts of substructure, which can dramatically change the number of chaotic orbits in a given potential; the amount of chaos is therefore sensitive to the physical nature of dark matter. This result motivates developing a quantitative framework to use the observed density structure of streams and shells to map the orbit structure of the host potentials. Critically, this would be applicable to imaging data where only the configuration-space morphology of the structures are observable, but can be tested and calibrated with the existing thin streams around the Milky Way.

% Ophiuchus
In Chapter \ref{ch:chaos-ophiuchus}, we present dynamical models for the newly-discovered and peculiarly-short Ophiuchus stream that suggest that the stream has formed in a strongly chaotic region of phase-space. We found that the stream's proximity to the Galactic center suggests that the bar must have a significant influence on its dynamical history: the triaxiality and time-dependence of the bar generates many chaotic orbits in the vicinity of the stream. We model the formation of the stream in a Milky Way potential model that includes a rotating bar and found that in all choices for the rotation parameters of the bar, orbits fit to the stream are strongly chaotic. Mock streams generated along these orbits qualitatively match the observed properties of the stream: because of chaos, stars stripped early generally form low-density, high-dispersion ``fans'' leaving only the most recently disrupted material detectable as a strong over-density. Our models predict that there should be more low-surface-brightness tidal debris than detected so far, likely with a complex phase-space morphology. This is the first time that chaos has been used to explain the properties of a stellar stream and is the first demonstration of the dynamical importance of chaos in the Galactic halo.

\section{Future Work} \label{sec:future-work}

\subsection{\rewinder}

\rewinder\ and other stream-modeling methods have demonstrated that they can \emph{precisely} infer potential parameters from simulated data where the analytic form of the gravitational potential is known but the parameter values are not known. This is, of course, a far reach from the real universe where the form of the potential is likely very complex. In our view, two critical enhancements must be made before \rewinder\ or any other stream-modeling methods can be meaningfully applied to observational data. First, they must use flexible or non-parametric forms for the potential, for example, using basis function expansions of the potential. This will make the potential inferences less precise but will be less biased. Second, these methods must simultaneously incorporate data from multiple streams.

% [What tracers are most informative]

\subsection{Chaos}

\subsubsection{Milky Way}

The frequency diffusion rate (FDR) is a measure of how fast the fundamental frequencies of a given orbit change due to dynamical chaos. An ensemble of orbits with a small spread in energy or frequencies (e.g., tidal debris from a low-mass progenitor) will have similar FDRs, and thus the FDR of the progenitor orbit correlates with the rate of density evolution of the debris \citep{apw15-chaos}. To convert from an observed rate of density evolution measured from a LSB image of substructure around external galaxies to a constraint on the FDR we will need to first measure the scaling relation between these quantities. In previous work, I computed the FDRs for a grid of orbits at a single energy level in a particular triaxial potential (Fig.~\ref{fig:fdiffmap}); for this work, I will use this same methodology to study grids of orbits in a variety of potential forms at many energy levels to measure the empirical relation between FDR and the density evolution of tidal debris.

Do the morphologies of thin streams around the MW constrain the triaxiality or shape of the Galactic potential? The light regions of the orbit grid shown in Fig.~\ref{fig:fdiffmap} are regular orbits at this energy surface. The thin streams around the MW must exist in such regions in the Galactic potential. I will run a suite of $N$-body simulations of stream formation along random orbits in a triaxial galaxy potential and use the final phase-space morphologies of the debris to measure the FDR for each stream (using the machinery proposed in the previous section). From this, I will learn how many streams are needed to effectively map the orbit structure of a potential to place interesting constraints on the global potential. I will then use the thin streams around the MW to independently constrain the shape of the halo to compare with the measurement from \texttt{Rewinder}.

Stream dispersal -- mixing in inner galaxy -- already dispersed streams??

\subsubsection{External}

Opportunity to use density alone to constrain statistical properties of halos???

The high-latitude survey (HLS) of the WFIRST mission will cover 2200 deg$^2$ with an unprecedented limiting $J$-band magnitude of $J=26.7$ \citep{spergel15}. With the expected surface brightness limits (32 mag arcsec$^{-2}$), WFIRST is expected to measure 5--10 distinct debris structures around each of over 100 of these nearby galaxies \citep{johnston08}. The surface-brightness limits of the upcoming LSST (29 mag arcsec$^{-2}$) are less sensitive, but the survey will contain a much larger sample of galaxies ($\approx$$10^6$). With large samples of LSB images of nearby galaxies in mind, I propose to develop new theoretical frameworks to use these data to infer the distribution of DM halo shapes and accretion histories. This proposed work is split into two complimentary approaches: (1) to develop a physical model for the expected distribution of morphologies of tidal debris structures given the halo shape and profile distributions (Sec.~\ref{sec:chaosmorphology}), and (2) to explore a data-driven method that uses statistical machine learning to infer these population properties (Sec.~\ref{sec:ml}).

Around large samples of nearby galaxies, we will not be able to measure velocities or distances to individual stream stars for the foreseeable future. Instead, deep surveys such as the the high-latitude survey of the \mission{WFIRST} mission will map stellar populations in 5--10 distinct stellar streams and shells around each of over 100 nearby galaxies \citep{johnston08,spergel15}. While the morphologies of streams will be more sensitive to chaos induced by triaxiality and intermediate-scale properties of dark matter halos, shells are more sensitive to the radial profile of the halo because the stellar orbits may come arbitrarily close to the center of the host galaxy where the profile is steepest and forces are strongest. Combining both types of debris structures will lead to better constraints on the host halo potentials: {\bf I will explore how chaotic frequency diffusion affects the density and morphology evolution of stellar shells in order to build a complete picture of how chaos alters the stellar halos of galaxies}.

Having statistics about dark matter halo properties will inform the search for dark matter: Different dark matter particle candidates (WIMPs, axions) predict vastly different amounts of substructure, which dramatically changes the number of chaotic orbits in a given potential. The amount of chaos is therefore sensitive to small-scale dark matter physics. The methods and insights developed here will be essential for using the deep imaging data provided from the \mission{WFIRST} and \mission{LSST} surveys to constrain dark matter models and cosmology.

\subsection{Ophiuchus}

Surprising that chaos matters: only 10s of orbits vs. 1000s of orbits .... ongoing searches for "fanned" stuff and tentative evidence.