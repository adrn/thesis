\chapter[Conclusion]{Conclusion}
\label{ch:conclusion}

Stellar streams are a unique tool for measuring the structure of mass around galaxies where the density of visible tracers is low. The work presented in this \article\ has shown that it will soon be possible to infer the 3D shape and distribution of dark matter around the Milky Way by modeling the many streams observed in the Galactic halo. This goal will be made possible by near-future catalogs of 6D kinematic measurements for individual stars in these streams from surveys like the \gaia\ mission. We have also shown that dynamical chaos can dramatically alter the morphological evolution of streams. Since the amount and significance of chaos depends strongly on the Galactic gravitational potential, this supports developing a new method for constraining the potential based purely on the morphologies of the many long, thin streams presently seen around the Milky Way. In this section, we summarize and discuss the key results presented in this \article\ and discuss future directions for this work.

\section{Summary of Results} \label{sec:summary}

% Rewinder 1
In Chapter \ref{ch:rewinder1}, we developed a new algorithm for using precise kinematic measurements of stars in stellar streams to measure the parameters of the underlying host galaxy's gravitational potential. This work was motivated by the prospect of combining extremely precise distance measurements for RR Lyrae stars with proper motion measurements from the \gaia\ mission and ground-based radial velocities to obtain full-space kinematics for samples of stars in stellar streams in the Galactic halo. The algorithm operates with the simple assumption that stars observed in a stellar stream were once part of the same progenitor system; by integrating the orbits of the progenitor system and stream stars backwards in time, in more correct host galaxy potential models the stars should come closer (in 6D phase-space) to the progenitor. We demonstrated that by modeling observations of a simulation of the Sagittarius stream with realistic but optimistic uncertainties, the true potential parameters can be inferred with small uncertainties and negligible biases. While this is encouraging and demonstrates the power of using dynamically cold structures for dynamical inference, it does is not a proper likelihood function and therefore cannot be used when there are missing or poorly measured data dimensions.

% Rewinder 2
In Chapter \ref{ch:rewinder2}, we used the ideas presented in Chapter~\ref{ch:rewinder1} to develop a probabilistic model for stellar streams. The key enhancement of this new approach is that it is written as a likelihood function with priors on the parameters so that missing or poorly-measured data can be incorporated into the inference of the potential parameters. \rewinder\ is distinct from many other stream-modeling methods because it (1) makes no assumption about the underlying form or integrability of the host galaxy potential, (2) non-parametrically infers the mass-loss history of the progenitor system, and (3) models the distribution function of stars \emph{at the time of stripping} rather than at present-day and can therefore analytically evaluate the likelihood without numerically reconstructing the density field. We tested \rewinder\ with a simulation observed with different assumptions about the tracer population (different data qualities) and showed that in all cases, all input potential parameters are successfully recovered: mass and length scaling but also the shape parameters of the input triaxial halo.

% Chaos / stream-morphology
In Chapter \ref{ch:chaos-morphology}, we showed that even when chaos is not important for restructuring the global orbit structure of a galaxy, chaos can greatly enhance the density evolution of stellar streams over just a few orbital times. This suggests that the morphology of tidal streams \emph{alone} can constrain the significance of chaos along the orbits of the progenitor systems, thereby placing constraints on the global properties of the gravitational potential. For example, different dark matter particle candidates (WIMPs vs. axions) predict vastly different amounts of substructure, which can dramatically change the number of chaotic orbits in a given potential; the amount of chaos is therefore sensitive to the physical nature of dark matter. This result motivates developing a quantitative framework to use the observed density structure of streams and shells to map the orbit structure of the host potentials. Critically, this would be applicable to imaging data where only the configuration-space morphology of the structures are observable, but can be tested and calibrated with the existing thin streams around the Milky Way.

% Ophiuchus
In Chapter \ref{ch:chaos-ophiuchus}, we present dynamical models for the newly-discovered and peculiarly-short Ophiuchus stream that suggest that the stream has formed in a strongly chaotic region of phase-space. We found that the stream's proximity to the Galactic center suggests that the bar must have a significant influence on its dynamical history: the triaxiality and time-dependence of the bar generates many chaotic orbits in the vicinity of the stream. We model the formation of the stream in a Milky Way potential model that includes a rotating bar and found that in all choices for the rotation parameters of the bar, orbits fit to the stream are strongly chaotic. Mock streams generated along these orbits qualitatively match the observed properties of the stream: because of chaos, stars stripped early generally form low-density, high-dispersion ``fans'' leaving only the most recently disrupted material detectable as a strong over-density. Our models predict that there should be more low-surface-brightness tidal debris than detected so far, likely with a complex phase-space morphology. This is the first time that chaos has been used to explain the properties of a stellar stream and is the first demonstration of the dynamical importance of chaos in the Galactic halo.

\section{Future Work} \label{sec:future-work}

\subsection{\rewinder}

\rewinder\ and other stream-modeling methods have demonstrated that they can \emph{precisely} infer potential parameters from simulated data where the analytic form of the gravitational potential is known but the parameter values are not known. This is, of course, a far reach from the real universe where the form of the potential is likely very complex. In our view, two critical enhancements must be made before \rewinder\ or any other stream-modeling methods can be meaningfully applied to observational data. First, they must use flexible or non-parametric forms for the potential, for example, using basis function expansions of the potential. This will make the potential inferences less precise but will be less biased. Second, these methods must simultaneously incorporate data from multiple streams.

% [What tracers are most informative]

\subsection{Chaotic stream dispersal as a potential constraint}

Does the existence of thin streams around the Milky Way constrain the triaxiality or shape of the Galactic potential? It is possible to rule out certain potentials that cause the observed thin streams to spread too quickly with their configuration-space morphology alone \citep[e.g.][]{pearson15}, but, of course, modeling the streams directly will provide more precise constraints on the potential. Instead, it is worth considering whether this could be used to measure the properties of a sample of external galaxy halos where it is not possible to obtain velocity measurements for individual tracers but streams can be observed with low-surface-brightness imaging. For example, the high-latitude survey of the WFIRST mission will cover 2200 deg$^2$ with an unprecedented limiting $J$-band magnitude of $J=26.7$ \citep{spergel15}. With the expected surface brightness limits (32 mag arcsec$^{-2}$), WFIRST is expected to measure 5--10 distinct debris structures around each of over 100 of these nearby galaxies \citep{johnston08}. The surface-brightness limits of the upcoming LSST (29 mag arcsec$^{-2}$) are less sensitive, but the survey will contain a much larger sample of galaxies ($\approx$$10^6$). In order to use these data, we must first develop a way to model the expected number and distribution of morphologies of tidal debris structures given dark matter halo shape and profiles. For example, we could study the tidal debris mixing rates as a function of frequency diffusion rate for a sample of halo shapes and profiles and use this to understand how many thin structures are expected to survive as a function of halo geometries. Though it is unclear whether this is possible in practice, it is an entirely new direction for research and would be the first proposed method for (statistically) measuring the 3D dark matter halo shapes for large samples of galaxies.

%\subsubsection{To understand the Galactic accretion history}
%
%If there are many chaotic orbits in the inner galaxy (near the bar) and in the Galactic halo, how does
%
%Stream dispersal -- mixing in inner galaxy -- already dispersed streams??

%\subsection{Ophiuchus}
%
%The Ophiuchus stream
%
%Surprising that chaos matters: only 10s of orbits vs. 1000s of orbits .... ongoing searches for "fanned" stuff and tentative evidence.
