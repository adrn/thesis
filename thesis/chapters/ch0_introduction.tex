\chapter[Introduction]{Introduction}
\label{ch:intro}

% I have seen the dark universe yawning,
%          Where the black planets roll without aim;
% Where they roll in their horror unheeded, without knowledge or lustre or name. - Lovecraft

% All you really need to know for the moment is that the universe is a lot more complicated than you might think, even if you start from a position of thinking it's pretty damn complicated in the first place. - Douglas Adams

% In the beginning the Universe was created.
% This has made a lot of people very angry and been widely regarded as a bad move. - Douglas Adams

\vspace{-16pt}
\begin{chapquote}{Douglas Adams}
\singlespacing
There is a theory which states that if ever anyone discovers exactly what the Universe is for and why it is here, it will instantly disappear and be replaced by something even more bizarre and inexplicable. There is another theory which states that this has already happened.
\end{chapquote}
\vspace{-8pt}
\noindent\makebox[\linewidth]{\rule{0.5\textwidth}{0.5pt}}
\vspace{1pt}

To many, the Milky Way represents the faint, fuzzy band of light that arcs across a dark, moonless sky.\footnote{But not to anyone who lives in New York City, where the majority of the research presented in this \article\ was conducted.} Long suspected to be a collection of unresolved stars (as early as the 1200s), Galileo Galilei confirmed this idea using the first astronomical telescopes in the 1600s. By 1755, Immanuel Kant speculated that the Milky Way may appear thin across the sky because we are viewing a large disk of stars from within, like a scaled-up version of the solar system with many more bodies (stars). Between then and the early 1900s, hundreds of ``spiral nebulae'' were discovered that were initially believed to be either star-forming regions within the Milky Way or external ``island universes.'' It took Edwin Hubble's distance measurements to the nearby galaxies M31 and M33---enabled by the concerted efforts of many astronomers---to clearly place them outside of the Milky Way and thus place our own Galaxy in the context of the universe of other galaxies.

Since the mid-1900s, through decades of research studying the structure of stars and gas at ever-increasing distances, we now know a great deal more about the Milky Way. Its scales are vast: the Galaxy hosts at least 100 billion stars over a distance of about 3 billion billion kilometers and took about 13 billion years to form to present day. But even more surprising than what we have observed is what we haven't: from the motion of gas in the outer Galaxy, the velocities of stars in external galaxies, and the gravitational lensing of distant galaxies, we now know that---by mass---the dominant substance in the universe is unseen, \emph{dark matter}.

Constraints on cosmological parameters from many independent sources---e.g., the cosmic microwave background \citep{planck15}, Type Ia supernova distances \citep{riess98, perlmutter99}, and baryon acoustic oscillations \citep{eisenstein05}---have found that the energy density of the universe is dominated by dark energy ($\approx$73\%) and dark matter ($\approx$23\%). Ordinary baryonic matter constitutes a meager 4\% of this energy density and at present is the only observable tracer of the vastly more numerous dark matter \citep[though the search for the dark matter particle is underway;][]{aprile11,luxdm12}. Dark matter dominates the mass of galaxies on large scales and choreographs their interactions. Therefore, the basis for expanding our understanding of the universe will come through characterization and detection of dark matter. %Some of the most pertinent questions of our time are: What is the nature of dark matter? How does the milky way fit in to cosmology / galaxy formation?]
The goal of the work presented in this \article\ is to develop methods to further our understanding of the global potential of the Milky Way in order to connect the Galaxy to cosmological predictions and---for the first time---precisely measure the structure of dark matter around a galaxy.

This work deals with the dynamics of structures in the farthest reaches of the Milky Way and [...]. In what follows, Galactocentric quantities are generally shown with no subscript, e.g., spherical radius $r$, cylindrical radius $R$. Heliocentric quantities use the [...] symbol in subscript---e.g., radial distance from the sun $r_\odot$. Capital, italicized galaxy components such as \mwdisk, \mwbulge, and \mwhalo\ refer to components of the Milky Way, whereas in lower-case these words refer to the components in general. 

\section{The Milky Way}\label{sec:milkyway}

In its global properties, the Milky Way is a fairly typical disk galaxy. It is, however, the one galaxy for which it is presently possible to measure detailed chemical abundances and full-space positions and velocities for large samples of individual stars (see Section~\ref{sec:surveys} for an overview of surveys that are measuring or will measure these chemo-dynamical quantities for stars in the outer Galaxy). This is of great interest to modeling the formation and evolution of the Galaxy: while a given star is born with a unique, frozen-in chemical ``tag'' apparent in its surface chemical abundances, the kinematics of the star will evolve from its birth location (in phase-space) due to dynamical processes such as radial migration, mixing, and heating. By comparing the observed spatial structure, kinematics, and chemical abundance patterns of stars in the Galaxy with hydrodynamical simulations, it is therefore hoped that these will provide strong constraints on galaxy formation mechanisms. 

Though many fundamental aspects of the formation of the Galaxy are still not understood\footnote{For example, were disk stars with large velocity dispersion born in that state or were their orbits heated?}, recent advancements have led to great strides in understanding of all of the major components of the Galaxy: the \mwdisk, \mwbulge, and \mwhalo. These components are briefly reviewed below.

\subsection{Disk}

The most massive baryonic component of the Milky Way is its disk. The mass in the \mwdisk\ \citep[$M_d \approx 5 \times 10^{10}~\msun$;][]{mcmillan11} is dominated by stars which have an approximately exponential density profile in both the radial and vertical directions \citep[with scale lengths of $\approx$2--3 kpc and $\approx$250--800 pc, respectively, from thin to thick disk;][]{ojha01, juric08, mcmillan11, bovy12-spatialMAP}. There is an appreciable amount of gas, the majority of which is present in a clumpy disk of neutral Hydrogen, HI \citep[$M_{\rm HI} \approx 8 \times 10^9~\msun$;][]{kalberla09} with a larger radial scale length ($\approx$3--4 kpc) and radially increasing scale height \citep[$\approx$100 pc at $R=8~\kpc$ to $\approx$1 kpc at $R=25~\kpc$;][]{wouterloot90, merrifield92}. However, outside of $R \gtrsim 12.5~\kpc$ the gas disk is warped up to $\approx$5 kpc away from the stellar midplane \citep{henderson82, kalberla07}. 

Within $R\approx 12~\kpc$, the HI and younger, more metal-rich stars orbit the Galaxy on approximately circular orbits with circular velocities $v_c \approx 220~\kms$, velocity dispersions $\sigma_v \approx 25~\kms \ll v_c$, and relatively small vertical excursions.\footnote{Stars in the thin disk have typical azimuthal orbital periods of $T_\phi \approx 200$--300 Myr. A typical star of the sun's age, $\approx$5 Gyr, has only completed $\approx$20 orbits around the Galaxy, in stark contrast with the billions of orbits the Earth has completed around the Sun.} Older and more metal-poor stars tend to have larger velocity dispersions and larger scale-heights. These sub-components of the \mwdisk\ are commonly referred to as the ``thin'' and ``thick'' \mwdisk s, respectively \citep{gilmore83}, though more recent work that has instead argued that the disk is more naturally decomposed into \emph{mono-abundance populations} that form a continuum of spatially super-imposed populations from younger, alpha-poor, and vertically compact to older, alpha-enhanced, and vertically extended \citep[see, e.g., Figure~12 and Section~6 in][]{rixbovy13, bovy12-nothickdisk}. Classically, the thin \mwdisk\ was believed to truncate around $R \approx 15~\kpc$ \citep[e.g.,][]{robin92}, but recent studies of the outer disk suggest that \mwdisk\ may extend farther and oscillate above and below the projected midplane measured in the inner Galaxy \citep{xu15, apw15-triand}. There could therefore be a significant number of \mwdisk\ stars at $15~\kpc \lesssim R \lesssim 40~\kpc$ with heights $|z| \gtrsim 1~\kpc$, but it is unclear how this connects to the midplane of the disk because of foreground dust extinction. In comparison, gas associated with the HI disk has been detected out to Galactocentric radii $R \approx 60~\kpc$ \citep{kalberla08}.

In the opposite direction---towards the inner Galaxy---dust extinction severely hampers studies of the Galaxy. Only the brightest and reddest stars observable in the infrared can pierce through the thick veils of interstellar dust associated with the gas in the thin \mwdisk. For this reason, while there is evidence for the existence of non-axisymmetric disk features along particular sight lines in multiple tracers \citep[e.g.,][]{levine06, reid14}, there is still no consensus on even the number of spiral arms in the \mwdisk\ let alone a global picture of their structure. Young stars in the thin disk within $R \lesssim 2~\kpc$ are almost entirely extincted. Fortunately, the \mwbulge\ contains a significant number of old, metal-rich giant stars that have recently enabled precise reconstructions of the stellar density and kinematics of bulge stars from $0~\kpc \lesssim R \lesssim 2~\kpc$.

\subsection{Bulge \& Bar}

The Milky Way \mwbulge\ is characterized by its predominantly old, metal-rich stellar population \citep[$t_{\rm age} \gtrsim 5~{\rm Gyr}$, ${[{\rm Fe}/{\rm H}]} \gtrsim -0.25$;][]{todo}, its smaller physical size \citep[$R \lesssim 1$--$2~\kpc$;][]{todo}, and its large velocity dispersions relative to the \mwdisk\ \citep[$\sigma_v \sim 100~\kms$;][]{todo}. There has long been evidence from, e.g., its boxy shape and from anomalous neutral gas motions near the Galactic center that the \mwbulge\ is a ``pseudobulge'' rather than a spherical ``classical'' bulge, and that there is likely a barred structure in the central Galaxy \citep{blitz91, binney91, weiland94, binney97}. 

Recent work has developed a convincing and complete picture of the barred \mwbulge. What is typically attributed to the cylindrically-rotating, boxy \mwbulge\ ($R \lesssim 2~\kpc$) is likely the inner component of a much longer bar that could extend up to 5 kpc from the Galactic center \citep{wegg13, wegg15}. The extension of the boxy \mwbulge\ into the disk is referred to as the ``long bar'' and is thinner in both vertical height and along the line of sight. Young, thin disk stars do extend farther inwards and appear to form a thin, barred disk parallel to but vertically compact relative to the boxy bulge \citep{dekany15}. Figure~XX shows the stellar density of \todo{XXX} inferred from the distribution of nearly 10 million red clump giant stars (RCGs) in the central Milky Way.\footnote{RCGs are evolved, helium-burning giant stars that have a small scatter in absolute magnitude and are therefore useful distance indicators.} The inner bar component is triaxial with exponential scale lengths $(h_x, h_y, h_z) = (0.70, 0.44, 0.18)~\kpc$, whereas the long bar component extends out to $R\approx 5~\kpc$ and is much flatter with scale lengths $(h_x, h_y, h_z) \approx (3.0, 0.7, 0.1)~\kpc$ \citep{wegg15}. 

The total mass in the \mwbulge\ is measured to be $M_b \approx 1.5$--$3 \times 10^{10}~\msun$ from both stellar population modeling \citep{dwek95, valenti15} and dynamical mass measurements \citep{zhao94, portail15}; the long bar contains $\approx$10\% of this mass \citep{wegg15}. These models are consistent with the existence of an additional classical (spherical) component that could account for up to $\lesssim 25\%$ of the bulge mass. Though observational constraints on a classical bulge component are difficult because of dust extinction, there is some indication of more a spherical stellar distribution that is chemically distinct from the barred population \citep{ness13a,ness13b, todo}.

The mass of the barred \mwbulge\ is fairly well constrained, however, its kinematic properties are not well known. Bars are generally assumed to rotate like solid bodies\footnote{Otherwise they would not be so numerous and prominent in disk galaxies.} and can therefore be characterized by their pattern speed, $\Omega_b$. For the \mwbulge, an additional kinematic quantity is the present-day angle of the bar relative to the sun-Galactic center axis, $\phi_b$. Many different methods have been applied to inferring the pattern speed and bar angle and are largely inconsistent, however it is generally believed that $25 < \Omega_b < 60 \kmskpc$ and $20^\circ < \phi_b < 30^\circ$ \citep{dwek95, stanek97, debattista02, shen10, wegg13, cao13, wegg15, portail15}.

\todo{Should I remove some stuff from intro to ophiuchus section because of this summary?}

\subsection{Halo}\label{sec:mw-halo}

The \mwhalo\ here refers to anything outside of the \mwdisk\ or \mwbulge\ that is still gravitationally bound to the Milky Way. This includes the gaseous \mwhalo, the stellar \mwhalo, and the dark matter \mwhalo. The stellar \mwhalo\ is the only of these three sub-components that is directly observable: the presence of the dark matter \mwhalo\ that dominates the total mass of the Galaxy on large scales is inferred indirectly, and the gaseous halo is low-density and has thus far been studied using absorption line features in background sources \citep{miller13}. The gaseous \mwhalo---though important for mediating inflows and outflows of gas to and from the \mwdisk---is likely unimportant for the orbits of halo stars and will be largely ignored in this \article. % \citep[$M_{\rm h:g} \approx 10^{10}~\msun \ll $][]{blitz10, salem15}

The stellar \mwhalo\ contains a tiny fraction of the baryons in the Galaxy \citep[$\approx$1\%;][]{todo} but is of great importance for studying the global structure of the Milky Way and its history. First, the stellar populations in the halo provide key insights about the early history of the Galaxy and led to the realization that a significant portion of the stellar \mwhalo\ was formed \emph{hierarchically} from the disruption and subsequent mixing of dwarf galaxies \citep[e.g.,][]{searle72, todo}. Second, \mwhalo\ stars orbit the Galaxy where dark matter dominates the mass profile and can therefore be used to measure the properties of the dark matter \mwhalo. In the \mwdisk, it is difficult to measure the density of dark matter because it is degenerate with, e.g., the scale length and mass of the baryonic components \citep{todo many}. In the halo, the number of stellar tracers is smaller, but they therefore contribute little to the gravitational potential. Third, the dynamical times in the \mwhalo\ are long ($\approx$0.5--1 Gyr or longer) and significant kinematic substructure has not had enough time to phase-mix away \citep{helmi99}. In fact, a significant fraction \citep[$\approx$40--50\%;][]{bell08} of the stellar halo is likely associated with substructure in the form of tidal streams and shells formed from disrupted or disrupting dwarf galaxies and, to a lesser extent, globular clusters \citep[e.g.,][]{newberg02,majewski03,belokurov06}. As will be discussed in Chapters~\ref{ch:rewinder1}--\ref{ch:rewinder2}, tidal debris features are extremely powerful for dynamical modeling because they are kinematically cold. 

The stellar \mwhalo\ is generally split into the inner and outer \mwhalo\ based on the steepening of the spherically- or axisymmetrically-averaged density profile observed in many different tracer populations (e.g., RR Lyrae stars, red giant branch stars, blue horizontal branch stars). The \mwhalo\ density profile appears to follow a double power-law ($n(r) \propto r^{-\alpha}$) where the inner halo slope $\alpha_{\rm in} \approx 2$--3 and the outer halo slope $\alpha_{\rm out} \approx 4$--6 with a break radius of $r_{\rm break} \approx 20$--30 kpc \citep{watkins09, sesar10, deason11, sesar11, sesar13}. The kinematics of these populations have been used to measure the velocity dispersion profile of the stellar \mwhalo\ and have found an analogous [...]: the radial velocity dispersion appears to decline steeply from $\sigma_r \approx 150~\kms$ at $r\approx10~\kpc$ to $\sigma_r \approx 100~\kms$ at $r\approx20~\kpc$, after which the decline slows until it reaches $\sigma_r \approx 50~\kms$ at $r\approx100~\kpc$ \citep{battaglia05, xue08, brown10, deason12b, deason13}. Together, the density and velocity dispersion profiles of the stellar \mwhalo\ have been used to constrain the shape and mass of the dark matter halo, finding that the virial mass of the Milky Way is $M_{\rm vir} \approx 10^{12}~\msun$ and . 

Many of the dynamical measurements that use stellar \mwhalo\ stars as tracers fundamentally assume that the observed sample of stars are drawn from a random or at least well-mixed distribution function \citep[e.g.,][]{todo jeans}. However, the phase-space distribution of the stellar \mwhalo\ is intricate and clumpy: the breaks and abrupt changes in moments of the stellar \mwhalo\ distribution function may result from the presence of a few dominant, unmixed merger remnants such as the Sagittarius stream \citep[which contains almost as much stellar mass as the rest of the stellar \mwhalo\ combined;][]{todo}. The existence of substructure has been shown to bias mass inferences made with random-tracer methods by up to 20\% \citep{yencho06}.\footnote{This bias is likely smaller than the systematic uncertainties in the velocity and mass profile measurements, but these uncertainties are typically not included in constraints made by such methods.} 

A complimentary approach to measuring the dark matter distribution is to instead take advantage of the non-random nature of the Galaxy's stellar distribution and utilize the knowledge that stars in tidal debris structures were once all part of the same object. Such approaches can require orders of magnitude fewer tracers than a randomly sampled population to achieve comparable accuracy. One method is to simply fit orbits to observations of streams \citep[e.g.,][]{koposov10}. However, the assumption that debris traces a single orbit is incorrect \citep[see][]{johnston98,helmi99}: changes in orbital properties along debris streams can lead to systematic biases in measurements of the Galactic potential \citep{eyre09a,varghese11}. Instead, the observed phase-space density of the stream should be compared to models that fully simulate the density evolution of tidally stripped debris. This has been done using full $N$-body simulations \citep{todo, law10}, however this is computationally expensive and therefore significantly limits the exploration of parameter-space. This has led to the development of a considerable number of methods that simulate the formation of tidal streams without the need for $N$-body calculations by approximately modeling only the disruption process, either in phase-space \citep{varghese11, kuepper12, todo} or in angle-action coordinates \citep{sanders,todo}. In Chapters~\ref{ch:rewinder1}--\ref{ch:rewinder2}, we develop and introduce a new method for using stellar tidal streams to infer the underlying mass distribution that makes no assumptions about the form or integrability of the potential (as is required by angle-action methods) and properly handles uncertainties or missing data in the kinematics of the individual stream stars.

% \subsection{Summary}

% The Milky Way is more than the sum of a few distinct components: the \mwdisk, \mwbulge, and \mwhalo\ are coupled in highly nonlinear ways and each interact and influence one another. Data from current and near-future surveys will require and enable a more [...wide eyed] view of the Galaxy [us to view the galaxy in sum to fully ... the information content in these surveys]. A better understanding of the formation and evolution of the Milky Way will provide a precise baseline by which to compare and study galaxy evolution more generally. This will also enable critical tests of cosmological theories on scales where 

\section{The Milky Way in context}\label{sec:milkyway-context}

Though the composition of dark matter (DM) is still unknown, the large-scale properties of the universe are precisely characterized by the $\Lambda$ Cold Dark Matter ($\Lambda$CDM) cosmological model: fluctuations in the cosmic microwave background and the clustering of galaxies across hundreds of megaparsecs are fit to extreme precision with $\Lambda$CDM \citep{planck15, sanchez12}. On smaller scales, simulations of galaxy formation have not converged on a set of concrete predictions, however a number of consistencies do appear across a range of DM models and even with inclusion of baryonic physics: (1) the spherically-averaged density profiles of DM halos seem to follow a universal profile across a large dynamic range in mass, (2) DM halos are permeated with substructure on many scales, (3) are triaxial in shape, and (4) have shapes and orientations that vary with radius \citep{dubinski91, navarro96, jing02, kuhlen07, veraciro11}. 

Within this paradigm, the Milky Way has grown to its present-day size and shape through a combination of steady accretion of gas and dark matter from the cosmic web and, to a lesser extent, through the accretion and merging of $\approx$1000 small galaxies that have fed gas and stars to the \mwdisk\ and \mwhalo. Evidence of the former is difficult to measure directly [...todo], however gas flows into high-redshift galaxies...[cite cosmic web imager?]. There is, however, clear evidence of the latter: to date, $\approx$45 satellite galaxies have been discovered around the Milky Way with a large range in masses, Galactocentric distances, and at different stages of merging with the Galaxy. \todo{Figure~XX summarizes ...}.

The largest mass satellites are the pair of massive dwarf galaxies visible to the naked eye from the southern hemisphere, the Large and Small Magellanic Clouds (LMC/SMC). The LMC--SMC system has already deposited nearly $XX \times 10^{YY}~\msun$ of gas into the \mwhalo\ \citep{putman-todo}, removed through a combination of ram-pressure and tidal stripping \citep{salem-todo}. The system appears to be on its first infall to the Milky Way \citep{besla10} so their stars have not been significantly stripped, but after several more orbits around the Galaxy the stellar \mwhalo\ will be completely dominated by the $M_{\rm LMC+SMC} \approx XX \times 10^{YY}~\msun$ of stars from this system.\footnote{Compare this mass to the stellar mass in the halo today (Section~\ref{sec:mw-halo}).} 

[Many other dwarf spheroidal galaxies, some show tidal distortion]. [Then there is the Sgr dwarf and stream, already long into process of merging, but given long dynamical times will take 10s of Gyrs to fully mix]. 

A naive comparison to cosmological simulations [...] that---even after correcting for observational biases---this number is an order of magnitude lower than the number of subhalos expected around Milky Way-mass galaxies in dark-matter-only simulations \citep[$\sim$1000;][]{missing-satellites-todo}. Higher resolution simulations with baryonic physics appear to shrink this expected number to within a factor of a few of the observed number, however, [....]

Finding and modeling streams and substructure are important because they were accreted galaxies: precise number of satellites need to reconstruct ones that were destroyed (galactic archaeology?). [What streams are there -- table?]

Around Milky Way, can measure full kinematics for stellar stream stars, best chance to constrain cosmology. So, MW is one galaxy, but also special because of our vantage point.

YY globular clusters and ZZ globular cluster streams. Streams could be used to detect dark subhalos. 

\section{Surveys of the Milky Way halo}\label{sec:surveys}

Future and ongoing surveys will revolutionize our knowledge about the outer disk and halo.

\begin{itemize}
	\item Surveys that will get kinematics for stars in the halo (ongoing and future surveys)
	\item Revolution in distance precision from tracers like RR Lyrae and spectral methods (twins), revolution in transverse velocity (Gaia)
	\item table summarizing different tracers and surveys? e.g., Kathryn's table
	\item Types of substructure (shells, streams)
\end{itemize}

% However, galaxy formation and evolution depends on many coupled, nonlinear processes: even with a wealth of information---kinematics and chemistry of XX of stars \citep{todo}---much is still unknown about fundamental properties of the Galaxy such as, for example, the number and locations of spiral arms in the \mwdisk, the pattern speed, mass, and size of the bar in the central Galaxy, and the three-dimensional shape and profile of the Milky Way's dark matter halo. 

% Like the majority of stars in the Milky Way, the sun was formed in and continues to orbit within the disk of the Galaxy. Though the precise details of its formation are still unknown, it likely formed in an open cluster of $\approx$$10^3$ stars that quickly dissolved $\approx$5 billion years ago. At its present Galactocentric distance of $\approx$8 kpc, the sun completes one orbit around the Galaxy every $\approx$200--250 million years. The sun has only orbited the galaxy $\approx$20 times in stark comparison to the billions of orbits the Earth has completed around the sun. This illustrates an important detail in the study of galaxies: the dynamical times intrinsic to galaxies are significant relative to their ages.

% Around Milky Way, can measure full kinematics for stellar streams, halo stars, will provide a great way to test cosmological predictions. Can also study structure around external galaxies via lensing or rotation curve, but limited by tracers. Elliptical galaxies, can use globular clusters as tracers. But still can't measure 3D positions and velocities. Lensing but only recover mass profile in projection. 