\chapter[Introduction]{Introduction}
\label{ch:intro}

\noindent{\bf  A few paragraphs to start -- very general, low-level intro to astronomy?}

The most common unit of length in this thesis is the kiloparsec (kpc) -- meter stick for galaxies like the milky way. $1~{\rm kpc} \approx 3300~{\rm light years}$.

The work in this \article\ attempts to connect the Milky Way to our cosmological environment ... \footnote{It is somewhat amusing that the majority of the research presented in this \article\ was conducted from New York city, where the surface brightness of the Milky Way is several magnitudes fainter than the light pollution levels.}

\section{The Milky Way}\label{sec:milkyway}

In its global properties, the Milky Way is a fairly typical disk galaxy. It is, however, the one galaxy for which it is presently possible to measure detailed chemical abundances and full-space positions and velocities for large samples of individual stars (see Section~\ref{sec:surveys} for an overview of surveys that are measuring or will measure these chemo-dynamical quantities for stars in the outer Galaxy). This is of great interest to modeling the formation and evolution of the Galaxy: while a given star is born with a unique, frozen-in chemical ``tag'' apparent in its surface chemical abundances, the kinematics of the star will evolve from its birth location (in phase-space) due to dynamical processes such as radial migration, mixing, and heating. By comparing observations of the spatial structure, kinematics, and chemical abundances of stars in the Galaxy to hydrodynamical simulations, it is therefore hoped that [aggregate quantities like the abundance patterns] will provide strong constraints on galaxy formation mechanisms. Though [detailed structure and processes not known], much clearer view of the structure of the major sub-[///] of the Milky Way: the \mwdisk, \mwbulge, and \mwhalo. Of course, the Galaxy is not a simple superposition of these three components---each interact and influence the others---but it is useful to try to understand them independently before studying coupled phenomena. 

\subsection{Disk}

The most massive baryonic component of the Galaxy is the flattened, rotating disk (or \mwdisk)\footnote{Hereafter, capital, italicized galaxy components such as \mwdisk, \mwbulge, and \mwhalo\ refer to components of the Milky Way, whereas in lower-case these words refer to the components in general.} with a total mass of $M_d \approx XX \times 10^{10}~\msun$ \citep{todo}. Though the baryonic mass in the \mwdisk\ is dominated by stars, there is an appreciable amount of cold gas \citep[$YY \times 10^ZZ~\msun$;][]{todo} that is still forming stars at a rate of $\approx 1~\msun~{\rm yr}^{-1}$ \citep{todo}. 

The gas and younger, metal-rich stars orbit the Galaxy on approximately circular orbits with relatively small vertical excursions, whereas older and more metal-poor stars tend to have larger scale-heights. Classically, these sub-components of the \mwdisk\ are known as the ``thin'' and ``thick'' \mwdisk s, respectively \citep{todo}, though more recent work that has decomposed the stellar populations in the disk into mono-abundance populations

However, non-axisymmetric kinematic features are observed in cold gas \citep{todo} and young stars \citep[cepheids;][]{todo} and are likely signatures of spiral arms \citep{todo}.

Within the extent of the \mwdisk\, $\approx$ZZ of mass is in the form of dark matter. 

The Galaxy is not simply a sum of these independent components: ... However, it can be useful to think of ...

Like the majority of stars in the Milky Way, the sun was formed in and continues to orbit within the disk of the Galaxy. 

Though the precise details of its formation are still unknown, it likely formed in a loose cluster of XX--YY stars (that has since dissolved) $\approx$5 billion years ago. At its present Galactocentric distance of $\approx$8 kpc, the sun completes one orbit around the Galaxy every $\approx$200--250 million years. The sun has only orbited the galaxy $\approx$20 times in stark comparison to the billions of orbits the Earth has completed around the sun. This illustrates an important detail in the study of galaxies: the dynamical times intrinsic to galaxies are significant relative to their ages.

Work in a comment about how, in recent years, even fundamental aspects of disk (extent and truncation, symmetry) called into question. Modes responding to satellite perturbations.

\subsection{Bulge \& Bar}

\subsection{Halo}

Finish talking about accretion of stuff and formation of structures, connecting to cosmology


However, galaxy formation and evolution depends on many coupled, nonlinear processes: even with a wealth of information---kinematics and chemistry of XX of stars \citep{todo}---much is still unknown about fundamental properties of the Galaxy such as, for example, the number and locations of spiral arms in the \mwdisk, the pattern speed, mass, and size of the bar in the central Galaxy, and the three-dimensional shape and profile of the Milky Way's dark matter halo. 

This is in part because the data are of sufficient quality that the Galaxy can no longer be viewed or modeled as a simple superposition of independent components (e.g., \mwdisk\ + \mwbulge\ + \mwhalo)

[...] Simulations have shown that galaxy components are interdependent and [...] \citep{todo}

In recent years, as computational [...] and statistical inference methods have improved substantially, 

[Problems because galaxy treated as independent parts, try to explain independently -- inter-related -- shifting from studying each independently to now connection between all]

[Transition to three subsections by saying galaxy can be roughly split into distinct components. In reality, can't be modeled independently, but makes sense to focus on each...?]





\begin{itemize}
	\item Our place in the galaxy -- not too hot, not too cold
	\item Components of the galaxy -- separate subsections for disk, bulge / bar, halo
	\item Just one galaxy, but the one galaxy where we can measure precise kinematics for individual stars and have vast numbers of tracers
	\item Different ways to use stars in halo to measure mass, shape: random tracers (e.g., Jeans) vs. cold tracers (stream modeling)
	\item Evidence for dark matter and tentative clues to its structure (e.g., lensing, and conclusions all over the place for MW halo shape from streams)
	\item Can understand the MW to inform studies of galaxy evolution in general
\end{itemize}

\section{The Milky Way in context}\label{sec:milkyway-context}

\begin{itemize}
	\item MW satellites, LMC/SMC, M31, Local group, etc.
	\item MW is just one galaxy, and just one type of galaxy -- there are many different types of galaxies across a range of masses
	\item Brief intro to cosmology and standard lore for galaxy formation and evolution
	\item DM theories like $\Lambda$CDM predict specific geometries for DM halo structure. Can we measure this? What are the implications?
\end{itemize}

\section{Surveys of the Milky Way halo}\label{sec:surveys}

\begin{itemize}
	\item Surveys that will get kinematics for stars in the halo (ongoing and future surveys)
	\item Revolution in distance precision from tracers like RR Lyrae and spectral methods (twins), revolution in transverse velocity (Gaia)
	\item Types of substructure (shells, streams)
	\item Census of known dwarf galaxies, GCs, streams, shells, etc. -- what are their properties?
\end{itemize}

\noindent{\bf  Part 4: Dynamics intro (from beginning of chaos paper)?}
\noindent{\bf  Part 5: Statistics intro?}
