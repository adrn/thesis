\chapter[Introduction]{Introduction}
\label{ch:intro}

\noindent{\bf  A few paragraphs to start -- very general, low-level intro to astronomy?}

The most common unit of length in this thesis is the kiloparsec (kpc) -- meter stick for galaxies like the milky way. $1~\kpc \approx 3300~{\rm light years}$.

The work in this \article\ attempts to connect the Milky Way to our cosmological environment ... \footnote{It is somewhat amusing that the majority of the research presented in this \article\ was conducted from New York city, where the surface brightness of the Milky Way is several magnitudes fainter than the light pollution levels.}

\section{The Milky Way}\label{sec:milkyway}

In its global properties, the Milky Way is a fairly typical disk galaxy. It is, however, the one galaxy for which it is presently possible to measure detailed chemical abundances and full-space positions and velocities for large samples of individual stars (see Section~\ref{sec:surveys} for an overview of surveys that are measuring or will measure these chemo-dynamical quantities for stars in the outer Galaxy). This is of great interest to modeling the formation and evolution of the Galaxy: while a given star is born with a unique, frozen-in chemical ``tag'' apparent in its surface chemical abundances, the kinematics of the star will evolve from its birth location (in phase-space) due to dynamical processes such as radial migration, mixing, and heating. By comparing the observed spatial structure, kinematics, and chemical abundance patterns of stars in the Galaxy with hydrodynamical simulations, it is therefore hoped that these will provide strong constraints on galaxy formation mechanisms. 

Though fundamental aspects of its formation are still not understood\footnote{For example, were disk stars with large velocity dispersion born in that state or were their orbits heated?}, considering the Galaxy as a whole rather than as a superposition of independent components has led to a better understanding of each of the major sub-populations of the Galaxy: the \mwdisk, \mwbulge, and \mwhalo.\footnote{Hereafter, capital, italicized galaxy components such as \mwdisk, \mwbulge, and \mwhalo\ refer to components of the Milky Way, whereas in lower-case these words refer to the components in general.} Each of these components is reviewed briefly below.

\subsection{Disk}

The most massive baryonic component of the Milky Way is its disk. The baryonic mass in the \mwdisk\ \citep[$M_d \approx 5 \times 10^{10}~\msun$;][]{mcmillan11, todo} is dominated by stars which have an approximately exponential density profile in both the radial and vertical directions \citep[with scale lengths of $\approx$2--3 kpc and $\approx$200--400 pc, respectively; ][]{mcmillan11, bovy12-spatialMAP, todo}\footnote{Though these numbers depend strongly on the age of the stellar population being fit.}. There is an appreciable amount of  gas, the majority of which is present in a clumpy disk of neutral Hydrogen, HI \citep[$M_{\rm HI} \approx 8 \times 10^9~\msun$;][]{kalberla09} with a larger radial scale length ($\approx$3--4 kpc) and radially increasing scale height \citep[$\approx$100 pc at $R=8~\kpc$ to $\approx$1 kpc at $R=25~\kpc$;][]{wouterloot90, merrifield92}. However, outside of $R \gtrsim 12.5~\kpc$ the gas disk is warped up to $\approx$5 kpc away from the stellar midplane. 

Within $R\approx 12~\kpc$, the HI and younger, more metal-rich stars orbit the Galaxy on approximately circular orbits with circular velocities $v_c \approx 220~\kms$, velocity dispersions $\sigma_v \approx 25~\kms \ll v_c$, and relatively small vertical excursions. Older and more metal-poor stars tend to have larger velocity dispersions and larger scale-heights. These sub-components of the \mwdisk\ are commonly referred to as the ``thin'' and ``thick'' \mwdisk s, respectively \citep{todo}, though more recent work that has instead argued that the disk is more naturally decomposed into \emph{mono-abundance populations} that form a continuum of spatially super-imposed populations from younger, alpha-poor, and vertically compact to older, alpha-enhanced, and vertically extended \citep[see, e.g., Figure~12 and Section~6 in][]{rixbovy13}. Classically, the thin \mwdisk\ was believed to truncate around $R \approx 12~\kpc$ \citep{todo}, but recent studies of the outer disk suggest that \mwdisk\ may extend farther and oscillate above and below the projected midplane measured in the inner Galaxy \citep{xu15, apw15-triand}. There could therefore be a significant number of \mwdisk\ stars at $15~\kpc \lesssim R \lesssim 40~\kpc$ with heights $|z| \gtrsim 1~\kpc$, but it is unclear how this connects to the midplane of the disk because of foreground dust extinction. In comparison, gas associated with the HI disk has been detected out to Galactocentric radii $R \approx 60~\kpc$ \citep{kalberla08}.

In the opposite direction---towards the inner Galaxy---dust extinction severely hampers studies of the Galaxy. Only the brightest and reddest stars observable in the infrared can pierce through the thick veils of interstellar dust associated with the gas in the thin \mwdisk. For this reason, while there is evidence for the existence of non-axisymmetric disk features along particular sight lines in cold gas \citep{todo} and young stars \citep[cepheids;][]{todo}, there is still no consensus on even the number of spiral arms in the \mwdisk\ let alone a global picture of their structure. Young stars in the thin disk within $R \lesssim 2~\kpc$ are almost entirely extincted. Fortunately, the \mwbulge\ contains a significant number of old, metal-rich giant stars that have recently enabled precise reconstructions of the stellar density and kinematics of bulge stars from $0~\kpc \lesssim R \lesssim 2~\kpc$.

\subsection{Bulge \& Bar}

The Milky Way \mwbulge---like those seen in external galaxies---is characterized by its predominantly old, metal-rich stellar population \citep[ages of XX, typical Fe/H;][]{todo}, its smaller physical size \citep[$R \lesssim 1$--$2~\kpc$;][]{todo}, and its large velocity dispersions relative to the \mwdisk\ \citep[$\sigma_v \sim 100~\kms$;][]{todo}. Bulges are typically classified as being either more spherical, ``classical'' bulges or the more common disky, rotating, ``pseudo-bulges'' that appear to associated with (or relics of) bar formation and growth. There has long been evidence from XX and anomalous neutral gas motions near the Galactic center that the \mwbulge\ is a pseudobulge \citep{blitz91, weiland94, binney97} and that there is likely a barred structure in the central Milky Way \citep{binney91}. 

Recent work has developed a convincing and complete picture of the barred \mwbulge: what has been attributed to the cylindrically-rotating, ``boxy/peanut'' \mwbulge\ is likely the inner component of a much larger bar that could extend up to 5 kpc from the Galactic center \citep{wegg13, todo}. Figure~XX shows the stellar density of the inner bar component inferred from the distribution of nearly 10 million red clump giant stars (RCGs) in the central Milky Way.\footnote{RCGs are evolved, helium-burning giant stars that have a small scatter in absolute magnitude and are therefore useful distance indicators.} The inner bar component is triaxial with scale lengths $(h_R, h_?, h_z) \approx (XX,XX,XX)~\kpc$, whereas the long bar component extends out to $R\approx 5~\kpc$ and is much flatter with scale lengths

\subsection{Halo} 

Gas -- Magellanic stream $10^9~\msun$, adds a lot of gas but how much to disk?

Finish talking about accretion of stuff and formation of structures, connecting to cosmology


However, galaxy formation and evolution depends on many coupled, nonlinear processes: even with a wealth of information---kinematics and chemistry of XX of stars \citep{todo}---much is still unknown about fundamental properties of the Galaxy such as, for example, the number and locations of spiral arms in the \mwdisk, the pattern speed, mass, and size of the bar in the central Galaxy, and the three-dimensional shape and profile of the Milky Way's dark matter halo. 

This is in part because the data are of sufficient quality that the Galaxy can no longer be viewed or modeled as a simple superposition of independent components (e.g., \mwdisk\ + \mwbulge\ + \mwhalo)

[...] Simulations have shown that galaxy components are interdependent and [...] \citep{todo}

In recent years, as computational [...] and statistical inference methods have improved substantially, 

[Problems because galaxy treated as independent parts, try to explain independently -- inter-related -- shifting from studying each independently to now connection between all]

[Transition to three subsections by saying galaxy can be roughly split into distinct components. In reality, can't be modeled independently, but makes sense to focus on each...?]


Of course, the Galaxy is more than the sum of distinct components---each interact and influence the others---but it is useful to try to understand them independently before studying coupled phenomena. 



% ----
[Typical number of orbits / dynamical times in each component]

Like the majority of stars in the Milky Way, the sun was formed in and continues to orbit within the disk of the Galaxy. Though the precise details of its formation are still unknown, it likely formed in an open cluster of $\approx$$10^3$ stars that quickly dissolved $\approx$5 billion years ago. At its present Galactocentric distance of $\approx$8 kpc, the sun completes one orbit around the Galaxy every $\approx$200--250 million years. The sun has only orbited the galaxy $\approx$20 times in stark comparison to the billions of orbits the Earth has completed around the sun. This illustrates an important detail in the study of galaxies: the dynamical times intrinsic to galaxies are significant relative to their ages.

\begin{itemize}
	\item Our place in the galaxy -- not too hot, not too cold
	\item Components of the galaxy -- separate subsections for disk, bulge / bar, halo
	\item Just one galaxy, but the one galaxy where we can measure precise kinematics for individual stars and have vast numbers of tracers
	\item Different ways to use stars in halo to measure mass, shape: random tracers (e.g., Jeans) vs. cold tracers (stream modeling)
	\item Evidence for dark matter and tentative clues to its structure (e.g., lensing, and conclusions all over the place for MW halo shape from streams)
	\item Can understand the MW to inform studies of galaxy evolution in general
\end{itemize}

\section{The Milky Way in context}\label{sec:milkyway-context}

\begin{itemize}
	\item MW satellites, LMC/SMC, M31, Local group, etc.
	\item MW is just one galaxy, and just one type of galaxy -- there are many different types of galaxies across a range of masses
	\item Brief intro to cosmology and standard lore for galaxy formation and evolution
	\item DM theories like $\Lambda$CDM predict specific geometries for DM halo structure. Can we measure this? What are the implications?
\end{itemize}

\section{Surveys of the Milky Way halo}\label{sec:surveys}

\begin{itemize}
	\item Surveys that will get kinematics for stars in the halo (ongoing and future surveys)
	\item Revolution in distance precision from tracers like RR Lyrae and spectral methods (twins), revolution in transverse velocity (Gaia)
	\item Types of substructure (shells, streams)
	\item Census of known dwarf galaxies, GCs, streams, shells, etc. -- what are their properties?
\end{itemize}

