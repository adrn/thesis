\chapter[Introduction]{Introduction}
\label{ch:intro}

\noindent{\bf  A few paragraphs to start -- very general, low-level intro to astronomy?}

State up front: goal is to develop methods to understand global Galactic potential to connect to cosmological predictions.

The most common unit of length in this thesis is the kiloparsec (kpc) -- meter stick for galaxies like the milky way. $1~\kpc \approx 3300~{\rm light years}$.

Galactocentric quantities with no subscript, e.g., spherical radius $r$, cylindrical radius $R$, heliocentric as $r_\odot$

This work deals with the dynamics of structures in the outer regions of the Milky Way --- first summarize what we know about the structure of the Galaxy....

The work in this \article\ attempts to connect the Milky Way to our cosmological environment ... \footnote{The majority of the research presented in this \article\ was conducted from New York city, where the surface brightness of the Milky Way is several magnitudes fainter than the sky brightness from light pollution.}

\section{The Milky Way}\label{sec:milkyway}

In its global properties, the Milky Way is a fairly typical disk galaxy. It is, however, the one galaxy for which it is presently possible to measure detailed chemical abundances and full-space positions and velocities for large samples of individual stars (see Section~\ref{sec:surveys} for an overview of surveys that are measuring or will measure these chemo-dynamical quantities for stars in the outer Galaxy). This is of great interest to modeling the formation and evolution of the Galaxy: while a given star is born with a unique, frozen-in chemical ``tag'' apparent in its surface chemical abundances, the kinematics of the star will evolve from its birth location (in phase-space) due to dynamical processes such as radial migration, mixing, and heating. By comparing the observed spatial structure, kinematics, and chemical abundance patterns of stars in the Galaxy with hydrodynamical simulations, it is therefore hoped that these will provide strong constraints on galaxy formation mechanisms. 

\todo{weak paragraph}
Though fundamental aspects of its formation are still not understood\footnote{For example, were disk stars with large velocity dispersion born in that state or were their orbits heated?}, considering the Galaxy as a whole rather than as a superposition of independent components has led to a better understanding of each of these sub-populations of the Galaxy: the \mwdisk, \mwbulge, and \mwhalo.\footnote{Hereafter, capital, italicized galaxy components such as \mwdisk, \mwbulge, and \mwhalo\ refer to components of the Milky Way, whereas in lower-case these words refer to the components in general.} Each of these components is reviewed briefly below.

\subsection{Disk}

The most massive baryonic component of the Milky Way is its disk. The mass in the \mwdisk\ \citep[$M_d \approx 5 \times 10^{10}~\msun$;][]{mcmillan11} is dominated by stars which have an approximately exponential density profile in both the radial and vertical directions \citep[with scale lengths of $\approx$2--3 kpc and $\approx$250--800 pc, respectively, from thin to thick disk;][]{ojha01, juric08, mcmillan11, bovy12-spatialMAP}. There is an appreciable amount of gas, the majority of which is present in a clumpy disk of neutral Hydrogen, HI \citep[$M_{\rm HI} \approx 8 \times 10^9~\msun$;][]{kalberla09} with a larger radial scale length ($\approx$3--4 kpc) and radially increasing scale height \citep[$\approx$100 pc at $R=8~\kpc$ to $\approx$1 kpc at $R=25~\kpc$;][]{wouterloot90, merrifield92}. However, outside of $R \gtrsim 12.5~\kpc$ the gas disk is warped up to $\approx$5 kpc away from the stellar midplane \citep{henderson82, kalberla07}. 

Within $R\approx 12~\kpc$, the HI and younger, more metal-rich stars orbit the Galaxy on approximately circular orbits with circular velocities $v_c \approx 220~\kms$, velocity dispersions $\sigma_v \approx 25~\kms \ll v_c$, and relatively small vertical excursions.\footnote{Stars in the thin disk have typical azimuthal orbital periods of $T_\phi \approx 200$--300 Myr; a typical star of the sun's age---$\approx$5 Gyr---has only completed $\approx$20 orbits around the Galaxy.} Older and more metal-poor stars tend to have larger velocity dispersions and larger scale-heights. These sub-components of the \mwdisk\ are commonly referred to as the ``thin'' and ``thick'' \mwdisk s, respectively \citep{gilmore83}, though more recent work that has instead argued that the disk is more naturally decomposed into \emph{mono-abundance populations} that form a continuum of spatially super-imposed populations from younger, alpha-poor, and vertically compact to older, alpha-enhanced, and vertically extended \citep[see, e.g., Figure~12 and Section~6 in][]{rixbovy13, bovy12-nothickdisk}. Classically, the thin \mwdisk\ was believed to truncate around $R \approx 15~\kpc$ \citep[e.g.,][]{robin92}, but recent studies of the outer disk suggest that \mwdisk\ may extend farther and oscillate above and below the projected midplane measured in the inner Galaxy \citep{xu15, apw15-triand}. There could therefore be a significant number of \mwdisk\ stars at $15~\kpc \lesssim R \lesssim 40~\kpc$ with heights $|z| \gtrsim 1~\kpc$, but it is unclear how this connects to the midplane of the disk because of foreground dust extinction. In comparison, gas associated with the HI disk has been detected out to Galactocentric radii $R \approx 60~\kpc$ \citep{kalberla08}.

In the opposite direction---towards the inner Galaxy---dust extinction severely hampers studies of the Galaxy. Only the brightest and reddest stars observable in the infrared can pierce through the thick veils of interstellar dust associated with the gas in the thin \mwdisk. For this reason, while there is evidence for the existence of non-axisymmetric disk features along particular sight lines in multiple tracers \citep[e.g.,][]{levine06, reid14}, there is still no consensus on even the number of spiral arms in the \mwdisk\ let alone a global picture of their structure. Young stars in the thin disk within $R \lesssim 2~\kpc$ are almost entirely extincted. Fortunately, the \mwbulge\ contains a significant number of old, metal-rich giant stars that have recently enabled precise reconstructions of the stellar density and kinematics of bulge stars from $0~\kpc \lesssim R \lesssim 2~\kpc$.

\subsection{Bulge \& Bar}

The Milky Way \mwbulge\ is characterized by its predominantly old, metal-rich stellar population \citep[$t_{\rm age} \gtrsim 5~{\rm Gyr}$, ${[{\rm Fe}/{\rm H}]} \gtrsim -0.25$;][]{todo}, its smaller physical size \citep[$R \lesssim 1$--$2~\kpc$;][]{todo}, and its large velocity dispersions relative to the \mwdisk\ \citep[$\sigma_v \sim 100~\kms$;][]{todo}. There has long been evidence from, e.g., its boxy shape and from anomalous neutral gas motions near the Galactic center that the \mwbulge\ is a ``pseudobulge'' rather than a spherical ``classical'' bulge, and that there is likely a barred structure in the central Galaxy \citep{blitz91, binney91, weiland94, binney97}. 

Recent work has developed a convincing and complete picture of the barred \mwbulge. What is typically attributed to the cylindrically-rotating, boxy \mwbulge\ ($R \lesssim 2~\kpc$) is likely the inner component of a much longer bar that could extend up to 5 kpc from the Galactic center \citep{wegg13, todo}. The extension of the boxy \mwbulge\ into the disk is referred to as the ``long bar'' and is thinner in both vertical height and along the line of sight. Young, thin disk stars do extend farther inwards and appear to form a barred thin parallel to but vertically compact relative to the boxy bulge \citep{dekany15 todo}. Figure~XX shows the stellar density of \todo{XXX} inferred from the distribution of nearly 10 million red clump giant stars (RCGs) in the central Milky Way.\footnote{RCGs are evolved, helium-burning giant stars that have a small scatter in absolute magnitude and are therefore useful distance indicators.} The inner bar component is triaxial with exponential scale lengths $(h_x, h_y, h_z) = (0.70, 0.44, 0.18)~\kpc$, whereas the long bar component extends out to $R\approx 5~\kpc$ and is much flatter with scale lengths $(h_x, h_y, h_z) \approx (3.0, 0.7, 0.1)~\kpc$ \citep{wegg15}. 

The total mass in the \mwbulge\ is measured to be $M_b \approx 1.5$--$3 \times 10^{10}~\msun$ from both stellar population modeling \citep{dwek95, valenti15} and dynamical mass measurements \citep{zhao94, portail15}; the long bar contains $\approx$10\% of this mass \citep{wegg15}. These models are consistent with the existence of an additional classical (spherical) component that could account for up to $\lesssim 25\%$ of the bulge mass. Though observational constraints on a classical bulge component are difficult because of dust extinction, there is some indication of more a spherical stellar distribution that is chemically distinct from the barred population \citep{ness13a,ness13b, todo}.

The mass of the barred \mwbulge\ is fairly well constrained, however, its kinematic properties are not well known. Bars are generally assumed to rotate like solid bodies\footnote{Otherwise they would not be so numerous and prominent in disk galaxies.} and can therefore be characterized by their pattern speed, $\Omega_b$. For the \mwbulge, an additional kinematic quantity is the present-day angle of the bar relative to the sun-Galactic center axis, $\phi_b$. Many different methods have been applied to inferring the pattern speed and bar angle and are largely inconsistent, however it is generally believed that $25 < \Omega_b < 60 \kmskpc$ and $20^\circ < \phi_b < 30^\circ$ \citep{dwek95, stanek97, debattista02, shen10, wegg13, cao13, wegg15, portail15}.

\todo{Should I remove some stuff from intro to ophiuchus section because of this summary?}

\subsection{Halo} 

The \mwhalo\ here refers to anything outside of the \mwdisk\ or \mwbulge\ that is still gravitationally bound to the Milky Way. This includes the gaseous \mwhalo, the stellar \mwhalo, and the dark matter \mwhalo. The stellar \mwhalo\ is the only of these three sub-components that is directly observable: the presence of the dark matter \mwhalo\ that dominates the total mass of the Galaxy on large scales is inferred indirectly, and the gaseous halo is low-density and has thus far been studied using absorption line features in background sources \citep{miller13}. The gaseous \mwhalo---though important for mediating inflows and outflows of gas to and from the \mwdisk---is likely unimportant for the orbits of halo stars and will be largely ignored in this \article. % \citep[$M_{\rm h:g} \approx 10^{10}~\msun \ll $][]{blitz10, salem15}

The stellar \mwhalo\ contains a tiny fraction of the baryons in the Galaxy \citep[$\approx$1\%][]{todo} but is of great importance for studying the global structure of the Milky Way and its history. First, the stellar populations in the halo provide key insights about the early history of the Galaxy and led to the realization that a significant portion of the stellar \mwhalo\ was formed \emph{hierarchically} from the disruption and subsequent mixing of dwarf galaxies \citep[e.g.,][]{searle72, todo}. Second, \mwhalo\ stars orbit the Galaxy where dark matter dominates the mass profile and can therefore be used to measure the properties of the dark matter \mwhalo. In the \mwdisk, it is difficult to measure the density of dark matter because it is degenerate with, e.g., the scale length and mass of the baryonic components \citep{todo many}. In the halo, the number of stellar tracers is smaller, but they therefore contribute little to the gravitational potential. Third, the dynamical times in the \mwhalo\ are long ($\approx$0.5--1 Gyr or longer) and significant kinematic substructure has not had enough time to phase-mix away \citep{helmi99}. In fact, a significant fraction \citep[$\approx$40--50\%;][]{bell08} of the stellar halo is likely associated with substructure in the form of tidal streams and shells formed from disrupted or disrupting dwarf galaxies and, to a lesser extent, globular clusters \citep[e.g.,][]{newberg02,majewski03,belokurov06}. As will be discussed in Chapters~\ref{ch:rewinder1}--\ref{ch:rewinder2}, tidal debris features are extremely powerful for dynamical modeling because they are kinematically cold. 

The stellar \mwhalo\ is generally split into the inner and outer \mwhalo\ based on the steepening of the spherically- or axisymmetrically-averaged density profile observed in many different tracer populations (e.g., RR Lyrae stars, red giant branch stars, blue horizontal branch stars). The \mwhalo\ density profile appears to follow a double power-law ($n(r) \propto r^{-\alpha}$) where the inner halo slope $\alpha_{\rm in} \approx 2$--3 and the outer halo slope $\alpha_{\rm out} \approx 4$--6 with a break radius of $r_{\rm break} \approx 20$--30 kpc \citep{watkins09, sesar10, deason11, sesar11, sesar13}. The kinematics of these populations have been used to measure the velocity dispersion profile of the stellar \mwhalo\ and have found an analogous [...]: the radial velocity dispersion appears to decline steeply from $\sigma_r \approx 150~\kms$ at $r\approx10~\kpc$ to $\sigma_r \approx 100~\kms$ at $r\approx20~\kpc$, after which the decline slows until it reaches $\sigma_r \approx 50~\kms$ at $r\approx100~\kpc$ \citep{battaglia05, xue08, brown10, deason12b, deason13}. Together, the density and velocity dispersion profiles of the stellar \mwhalo\ have been used to constrain the shape and mass of the dark matter halo, finding that the virial mass of the Milky Way is $M_{\rm vir} \approx 10^{12}~\msun$ and . 

Many of the dynamical measurements that use stellar \mwhalo\ stars as tracers fundamentally assume that the observed sample of stars are drawn from a random or at least well-mixed distribution function \citep[e.g.,][]{todo jeans}. However, the phase-space distribution of the stellar \mwhalo\ is intricate and clumpy: the breaks and abrupt changes in moments of the stellar \mwhalo\ distribution function may result from the presence of a few dominant, unmixed merger remnants such as the Sagittarius stream\footnote{When the Large Magellanic Cloud (LMC) ultimately merges with the Milky Way, stars from the LMC will likely completely dominate the future stellar halo.} \citep[which contains almost as much stellar mass as the rest of the stellar \mwhalo\ combined;][]{todo}. The existence of substructure has been shown to bias mass inferences made with random-tracer methods by up to 20\% \citep{yencho06}.\footnote{This bias is likely smaller than the systematic uncertainties in the velocity and mass profile measurements, but these uncertainties are typically not included in constraints made by such methods.} 

A complimentary approach to measuring the dark matter distribution is to instead take advantage of the non-random nature of the Galaxy's stellar distribution and utilize the knowledge that stars in tidal debris structures were once all part of the same object. Such approaches can require orders of magnitude fewer tracers than a randomly sampled population to achieve comparable accuracy. One method is to simply fit orbits to observations of streams \citep[e.g.,][]{koposov10}. However, the assumption that debris traces a single orbit is incorrect \citep[see][]{johnston98,helmi99}: changes in orbital properties along debris streams can lead to systematic biases in measurements of the Galactic potential \citep{eyre09a,varghese11}. Instead, the observed phase-space density of the stream should be compared to models that fully simulate the density evolution of tidally stripped debris. This has been done using full $N$-body simulations \citep{todo, law10}, however this is computationally expensive and therefore significantly limits the exploration of parameter-space. This has led to the development of a considerable number of methods that simulate the formation of tidal streams without the need for $N$-body calculations by approximately modeling only the disruption process, either in phase-space \citep{varghese11, kuepper12, todo} or in angle-action coordinates \citep{sanders,todo}. In Chapters~\ref{ch:rewinder1}--\ref{ch:rewinder2}, we develop and introduce a new method for using stellar tidal streams to infer the underlying mass distribution that makes no assumptions about the form or integrability of the potential (as is required by angle-action methods) and properly handles uncertainties or missing data in the kinematics of the individual stream stars.

\section{The Milky Way in context}\label{sec:milkyway-context}

The dark matter \mwhalo\ of the Milky Way is only one, but only one we can see in 3D. Evidence for DM from lensing, rotation curves, but dark matter theories seem to consistently predict

Large-scale, cosmological simulations of galaxy formation in the $\Lambda$CDM paradigm suggest that the spherically-averaged density profiles of dark matter halos follow a universal profile across a large dynamic range in mass \citep{navarro96}. High resolution simulations --- both with and without baryons --- have produced dark matter halos that (1) are permeated with substructure on many scales, (2) are triaxial in shape, and (3) have shapes and orientations that vary with radius \citep{dubinski91, jing02, kuhlen07, veraciro11}. Dark-matter-only simulations produce triaxial halos \citep{jing02} with significant density fluctuations \citep{zemp09}. Inclusion of baryons tends to soften the triaxiality and graininess in the inner galaxy through a combination of dissipative infall \citep{dubinski94} or cooling \citep{bryan13}. These processes combined with the gravity from a baryonic disk or ellipsoid can act to make the inner halo more oblate or spherical, however they do not seem to erase the clumpy, triaxial nature of the outer halo \citep[e.g.,][]{pontzen12}. This can lead to radially-dependent axis ratios, orientation, and smoothness, and suggests that the true mass distributions around Milky Way-like galaxies are not easily represented by simple, time-independent potentials. Methods that seek to measure the gravitational potentials around such galaxies must be flexible enough to handle generic potential forms where finding simple analytic approximations or computing actions may not be possible.

The bulk of the baryonic matter in galaxies spans roughly 5-10\% of the spatial extent of the host dark matter halo. Hence, the brightest and most easily observable components of a galaxy are sensitive to the inner portion of the host halos mass distribution. For example, the rotation curves of disk galaxies trace the inner mass with exquisite sensitivity since matter in disks can be assumed to move on nearly circular orbits. Measurements of the dark matter distribution at large radii is complicated by the low density of visible tracers, observational difficulties of measuring kinematics of stars at large distances, and unknown orbits. Around external galaxies, the extended mass distribution has been studied using a variety of approaches \citep[see][for a complete and detailed review]{courteau13}. For example, the kinematics of tracer populations such as globular clusters or planetary nebulae can be used to derive mass estimates under the assumptions that these satellite systems are on random orbits and are well-mixed in orbital phase \citep[early investigations include][]{mendez01,cote03}. Simple, parameterized fits to both the mass and orbit distribution have been simultaneously constrained using such data \citep[e.g.][]{napolitano11,deason12c}. Alternatively, the statistical properties of gravitationally lensed background sources around a galaxy can be used to constrain the \emph{projected} shape, orientation, and radial profile of mass \citep[see, for example, the Lens Structure and Dynamics Survey described in][]{koopmans02}. Of course, lensing reconstructions can only be performed for galaxies which closely intersect our line of sight to background sources, but the advent of large photometric catalogues has allowed automatic searches for such chance alignments and significant increases in the number of objects studied in this way \citep[e.g. the Sloan Lens ACS Survey, see][]{bolton06}.

Within the Milky Way our unique vantage point allows us a three-dimensional view of stars within our own dark matter halo. Our proximity allows us to use individual stars as kinematic tracers and hence build much larger samples that probe deeper into the halo than the globular cluster and planetary nebula studies of external galaxies. For example, \cite{deason12a} used halo BHB stars selected from the Sloan Digital Sky Survey \cite[SDSS;][]{york00} as a random tracer population to measure the mass and slope of a power-law fit to the potential. Such studies assume that the tracer orbits are randomly sampled from a smooth distribution function and are fully phase mixed. However, large photometric surveys such as the SDSS and 2MASS \citep{skrutskie06} have discovered copious amounts of substructure --- in streams and kinematic associations of stars --- in the Milky Way halo \citep[e.g.,][]{belokurov06, rochapinto04}, thus demonstrating that the stellar distribution is neither on random orbits nor fully phase-mixed. Substructure in the form of stellar streams and clouds is known to bias mass and velocity inferences from random tracer methods by several tens of percent \citep{yencho06}.


\subsection{Satellites}

\begin{itemize}
	\item Evidence for dark matter and tentative clues to its structure (e.g., lensing, and conclusions all over the place for MW halo shape from streams)
	\item MW satellites, LMC/SMC, M31, Local group, etc.
	\item MW is just one galaxy, and just one type of galaxy -- there are many different types of galaxies across a range of masses
	\item Brief intro to cosmology and standard lore for galaxy formation and evolution
	\item DM theories like $\Lambda$CDM predict specific geometries for DM halo structure. Can we measure this? What are the implications?
\end{itemize}

\section{Surveys of the Milky Way halo}\label{sec:surveys}

\begin{itemize}
	\item Surveys that will get kinematics for stars in the halo (ongoing and future surveys)
	\item Revolution in distance precision from tracers like RR Lyrae and spectral methods (twins), revolution in transverse velocity (Gaia)
	\item table summarizing different tracers and surveys? e.g., Kathryn's table
	\item Types of substructure (shells, streams)
	\item Census of known dwarf galaxies, GCs, streams, shells, etc. -- what are their properties?
	\item How many accreted things? See XX dwarf galaxies, XX globular clusters, XX streams
\end{itemize}



% However, galaxy formation and evolution depends on many coupled, nonlinear processes: even with a wealth of information---kinematics and chemistry of XX of stars \citep{todo}---much is still unknown about fundamental properties of the Galaxy such as, for example, the number and locations of spiral arms in the \mwdisk, the pattern speed, mass, and size of the bar in the central Galaxy, and the three-dimensional shape and profile of the Milky Way's dark matter halo. 

% Like the majority of stars in the Milky Way, the sun was formed in and continues to orbit within the disk of the Galaxy. Though the precise details of its formation are still unknown, it likely formed in an open cluster of $\approx$$10^3$ stars that quickly dissolved $\approx$5 billion years ago. At its present Galactocentric distance of $\approx$8 kpc, the sun completes one orbit around the Galaxy every $\approx$200--250 million years. The sun has only orbited the galaxy $\approx$20 times in stark comparison to the billions of orbits the Earth has completed around the sun. This illustrates an important detail in the study of galaxies: the dynamical times intrinsic to galaxies are significant relative to their ages.


% The Milky Way is more than the sum of a few distinct components: the \mwdisk, \mwbulge, and \mwhalo\ are coupled in highly nonlinear ways and each interact and influence one another. Data from current and near-future surveys will require and enable a more [...wide eyed] view of the Galaxy [us to view the galaxy in sum to fully ... the information content in these surveys]. A better understanding of the formation and evolution of the Milky Way will provide a precise baseline by which to compare and study galaxy evolution more generally. This will also enable critical tests of cosmological theories on scales where 