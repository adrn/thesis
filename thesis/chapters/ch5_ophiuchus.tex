\chapter[Chaotic fanning of the Ophiuchus stream]{Chaotic fanning of the Ophiuchus stream}
\label{ch:chaos-ophiuchus}
% \let\thefootnote\relax\footnotetext{This section contains text from an article published in XXX. Portions of this paper's introduction now appear in the introduction to this dissertation.}

% TODO: do I need \usepackage[caption=false]{subfig}?
% \usepackage{multirow}

% TODO this could change after referee response...

\newcommand{\lyapexp}{\lambda_{\rm max}}
\newcommand{\lyapt}{t_\lambda}

%\title{Spending too much time at the Galactic bar: chaotic fanning of the Ophiuchus stream}
%\author{Adrian M. Price-Whelan\altaffilmark{\colum,\adrn},
%            Branimir Sesar\altaffilmark{\mpia},
%            Kathryn V. Johnston\altaffilmark{\colum},
%            Hans-Walter Rix\altaffilmark{\mpia}
%}
%
%% Affiliations
%\newcommand{\colum}{1}
%\newcommand{\adrn}{2}
%\newcommand{\mpia}{3}
%
%\altaffiltext{\colum}{Department of Astronomy,
%                      Columbia University,
%                      550 W 120th St.,
%                      New York, NY 10027, USA}
%\altaffiltext{\adrn}{To whom correspondence should be addressed: adrn@astro.columbia.edu}
%\altaffiltext{\mpia}{Max-Planck-Institut f\"ur Astronomie,
%                     K\"onigstuhl 17, D-69117 Heidelberg, Germany}
%
%\begin{abstract}
%The Ophiuchus stellar stream is peculiar: (1) its length is short given the age of its constituent stars, and (2) several probable member stars that lie close in both sky position and velocity have dispersions in these dimensions that far exceed those seen within the stream.
%The stream's proximity to the Galactic center suggests that the bar must have a significant influence on its dynamical history: The triaxiality and time-dependence of the bar may generate chaotic orbits in the vicinity of the stream that can greatly affect its morphology.
%We explore this hypothesis with models of stream formation along orbits consistent with Ophiuchus' properties in a Milky Way potential model that includes a rotating bar.
%We find that in all choices for the rotation parameters of the bar, orbits fit to the stream are strongly chaotic.
%Mock streams generated along these orbits qualitatively match the observed properties of the stream: because of chaos, stars stripped early generally form low-density, high-dispersion ``fans'' leaving only the most recently disrupted material detectable as a strong over-density.
%Our models predict that there should be more low-surface-brightness tidal debris than detected so far, likely with a complex phase-space morphology.
%The existence of or lack of these features around the Ophiuchus stream would provide an interesting constraint on the properties of the Milky Way bar and would help distinguish between formation scenarios for the stream.
%This is the first time that chaos has been used to explain the properties of a stellar stream and is the first demonstration of the dynamical importance of chaos in the Galactic halo.
%The existence of long, thin streams around the Milky Way---presumably formed along non- or weakly-chaotic orbits---may represent only a subset of the total population of disrupted satellites.
%\end{abstract}
%
%\keywords{
%  Galaxy: halo
%  ---
%  globular clusters: general
%  ---
%  stars: kinematics and dynamics
%  ---
%  Galaxy: structure
%  ---
%  Galaxy: kinematics and dynamics
%}

\section{Introduction}\label{sec:ch5-introduction}

The Ophiuchus stream \citep{bernard14, sesar15a} is a recently discovered
stellar tidal stream that sits above the Galactic bulge at a Galactocentric
radius and height $(R,z) \approx (1.5, 4.3)~{\rm kpc}$. All observational
evidence suggests that the stream is a completely disrupted globular cluster:
The stream stars have (1) a small positional dispersion orthogonal to the
extended direction of the stream (width $\approx$10 pc, length $\approx$1.5
kpc); (2) no detectable over-density along the stream that could be the
progenitor system; (3) a small velocity dispersion $\approx$0.4 ${\rm km}~{\rm
s}^{-1}$; and (4) an old stellar population ($\approx$12 Gyr) estimated from
isochrone fitting \citep[][hereafter S15]{sesar15a}.

There are a number of peculiarities about the observed kinematics of the
Ophiuchus stream. For example, the de-projected length of the visible part of
the stream is short given the age of its stellar population ($\approx$1.5 kpc).
S15 fit an orbit to the kinematics of the stream stars in a static, axisymmetric
model for the gravitational field of the inner Galaxy and ran N-body simulations
of globular clusters on this orbit. S15 find that---on this orbit---the portion
of the stream visible as an over-density in main-sequence stars must have been
formed in the last $\lesssim$400 Myr for the stream to remain as short as it is
observed. This dynamical age is at odds with the old ($\approx$10--12 Gyr)
stellar population: The abrupt end of the stream suggests that the cluster
apparently fully disrupted at once in the last 400 Myrs. Another puzzle is the
existence of blue horizontal branch (BHB) stars close to the stream (within a
few degrees) with similar radial velocities, but with a large dispersion in both
sky position and velocity \citep[][hereafter S16]{sesar16}. The stream has a
very distinct and large line-of-sight velocity ($\approx$290 \kms) and is
therefore easily detected above the background halo population. Four BHB stars
have been detected with line-of-sight velocities $>230~\kms$ that lie close to
an extrapolation of the stream on the sky. This makes them likely members of the
stream, as their velocities are in stark contrast to the background halo
population (see Section 4, S16). Yet, they have a velocity dispersion
$\approx$75 times larger than the measured internal velocity dispersion of the
stream stars. These stars hint at the existence of associated low-density,
high-dispersion features that were not modeled in S15 and are not predicted by
the $N$-body simulations from this prior work, which assume a sudden, total
disruption of the stream progenitor.

% ---------------------------------------------------------------------------------
\begin{figure*}[!tbp]
\begin{center}
%\includegraphics[width=0.7\textwidth]{figures/potentials-four}
\includegraphics[width=\textwidth]{figures/ch4/potentials-four}
\caption{{\it Left column:} Circular velocity curves along the Sun-Galactic
center line for a representative barred MW potential model (top left) and for
the static MW potential model (bottom left). Solid black line shows total (sum
of all components), lines below show a decomposition by potential components.
Vertical grey bar shows approximate position of the Sun, horizontal grey bar
shows roughly the range in measured circular velocity of the Sun. {\it Right
column:} Contours of constant surface density for a barred MW potential (top
right) and the static MW potential model (bottom right). Four contours are drawn
per decade in surface density between $10^7$ and $10^{12}~\msun~{\rm kpc}^{-3}$.
Note the perturbation from the bar potential within Galactocentric radius $r
\lesssim 4~{\rm kpc}$. The Sun's position is indicated by the `$\odot$' symbol.
}
\label{fig:potentials}
\end{center}
\end{figure*}

The orbit fit and $N$-body simulations in S15 used a static, axisymmetric
potential to represent the Milky Way potential, but it is well-known that the
Galactic bulge contains a triaxial, rotating, bar-like structure several kpc in
size \citep[e.g.,][]{blitz91, weinberg92, dwek95, wegg13}. Given the proximity
of the stream to the center of the Galaxy, the time-dependent, triaxial
potential of the Galactic bar must be taken into account when modeling the orbit
of the Ophiuchus stream. The presence of a bar-like perturbation to the
potential will change the orbit of the stream progenitor and the orbit structure
in the inner galaxy \citep{zotos12, portail15b, gajda15}. Bar-like features can
also introduce a significant number of chaotic orbits in their vicinity
\citep{weinberg15} and generate resonances that may also affect stream formation
\citep{hattori15}.

Recent work has shown that dynamical chaos can dramatically alter the density
evolution of tidal streams \citep[e.g.,][]{fardal14, apw15-chaos}. Along certain
chaotic orbits, the stream stars will spread much faster in 3D position than
from ordinary phase-mixing and, depending on the orbital phase at which the
stream is observed, may develop large, low-density ``fans'' of stars at the ends
of a stream \citep{pearson15, apw15-chaos}. As a first application of this
theoretical understanding, we study whether stream-fanning---chaotic or simply
from density evolution in a triaxial, time-dependent potential---could plausibly
explain the observed properties of the Ophiuchus stream. In particular, we
consider whether such models can reproduce:
\begin{enumerate}
	\item the apparent shortness and fast density truncation of the stream;
	\item the increased positional dispersion of the four new candidate members from S16;
	\item the large velocity dispersion of the S16 stars.
\end{enumerate}
We do not aim to perfectly represent the observed data, but rather to explore
the plausibility of explaining the peculiarities of the stream using chaotic
stream fanning. Note that an alternate model was recently proposed that instead
places the stream progenitor on an orbit in resonance with the bar
\citep{hattori15}. We discuss the differences between these two models in
Section~\ref{sec:ch5-discussion}.

In Section~\ref{sec:ch5-method} we describe the methods used in this work: in
Section~\ref{sec:ch5-potential} we describe the models we use for the gravitational
potential of the Galaxy, in Section~\ref{sec:ch5-orbitfit-nonapdx} we outline the
probabilistic procedure we use to fit orbits to the stream data (explained in
detail in Appendix~\ref{sec:ch5-orbitfit}), and in Section~\ref{sec:ch5-mocks} we
explain the simple method we use to generate mock streams. In
Section~\ref{sec:ch5-results1} we discuss the results from fitting orbits to the
data in a static, axisymmetric potential model and several potential models with
a time-dependent bar. In Section~\ref{sec:ch5-results2} we generate mock streams
along these orbits to argue that chaotic stream-fanning is a plausible
explanation for the observational peculiarities of the Ophiuchus stream. We
discuss the implications of this work and possibilities for future work in
Section~\ref{sec:ch5-discussion} and we conclude in Section~\ref{sec:ch5-conclusions}.

\section{Methods}\label{sec:ch5-method}

Our goal is to (1) assess whether the Galactic bar can produce chaotic orbits in
the vicinity of the Ophiuchus stream and (2) determine if chaotic density
evolution of tidal debris stripped from the progenitor of this stream can
explain the apparent shortness of the stream and low-density, high-dispersion
stars beyond the extent of main sequence stars observed in PS1. In this section,
we describe the potential models we use to represent the galaxy and outline the
methods we use to detect and quantify the strength of chaos for individual
orbits. We then describe the likelihood function we use for fitting orbits to
the stream stars. Finally, we describe how we generate mock stellar streams for
a progenitor on a given orbit.

Throughout we assume the Sun is at Galactocentric position $(x,y,z) =
(-8.3,0,0)~{\rm kpc}$ \citep[e.g.,][]{schoenrich12} with velocity $(v_x,v_y,v_z)
= (-11.1, 250, 7.25)~\kms$ \citep[e.g.,][]{schoenrich10, schoenrich12}.

\subsection{Potential models}\label{sec:ch5-potential}
To integrate orbits and to compute chaos indicators we must choose a
gravitational potential model to represent the potential of the Milky Way. The
key feature of the potential that we would like to capture is the
time-dependence and triaxiality of the Galactic bar. Recent work has used
stellar number counts of Red Clump giant stars in the Galactic bulge to
constrain dynamical models of the bar \citep{portail15}. Measurements of the
total mass of the bar feature from this study are largely consistent with past
work \citep[e.g.,][]{wang12}, however the measured pattern speed and present bar
angle are significantly discrepant and this difference is not fully understood.
We construct a parametrized potential model consisting of a triaxial,
time-dependent (rotating) bulge component added to simple models for the disk
and halo of the Milky Way. We describe below how we fix the parameters of the
disk, halo, and bar or bulge component, but explore different choices for the
time-dependence and orientation of the bar. We also define a static potential
with a spherical bulge for comparison.

These potential models are meant to be representative rather than definitive.
The uncertainty in the Milky Way potential within Galactocentric radii of
$r\lesssim 4~{\rm kpc}$ and outside of $r\gtrsim 15~{\rm kpc}$ are large enough
that trying to match the exact density distribution of the Ophiuchus stream is
not a useful exercise. Instead, we consider qualitatively different potentials
that allow us to isolate and study the affect of chaotic stream-fanning of tidal
debris in the vicinity of the stream.

\subsubsection{Barred potential}
We use a spherical Navarro-Frenk-White potential to represent the dark matter
halo \citep{navarro96} parametrized as
\begin{align}
	\Phi(r) &= -v_h^2\,\frac{\ln{(1 + r/r_s)}}{r/r_s}\label{eq:nfw}
\end{align}
and a Miyamoto-Nagai potential for the disk \citep{miyamoto75}. For the bar
component, we use a basis function expansion (BFE) of the potential and density
of the bar with expansion coefficients derived for a triaxial, exponential bar
density \citep[][hereafter W12]{wang12}. We use the pre-computed expansion
coefficients used in W12, which were computed from a low-order expansion of the
triaxial bar density used in \citet{dwek95}.\footnote{The coefficients presented
in W12 are for just the cosine terms (the $A_{lm}$ in \citet{hernquist92} or the
$S_{nlm}$ in \citet{lowing11}) because all sine terms have zero coefficients for
a triaxial density function.} We have implemented the BFE computation of the
potential, density, and gradient of the potential in \texttt{C} and \python\ and
the code is publicly available on
\github.\footnote{\url{https://github.com/adrn/biff}}

The BFE representation fixes the axis ratios of the bar---that is, the
exponential scale lengths along the three axes of the bar were adopted from
\cite{dwek95} when the expansion coefficients were calculated in W12; all other
potential parameter values are given in Table~\ref{tbl:potential-params-barred}.
The mass of the halo is fixed and the mass of the disk and bar are varied in
order to qualitatively reproduce the flatness and amplitude of the circular
velocity curve of the Milky Way \citep{bovy12-circvel}.
Figure~\ref{fig:potentials}, top left shows the circular velocity along the line
connecting the Sun to the Galactic center in this model (the Galactic $x$ axis).
Figure~\ref{fig:potentials}, top right shows contours of constant surface
density for a face-on (left) and edge-on (right) view of this potential model
with the bar angle set to $20^\circ$ \citep[compare to, e.g., Figure 3
in][]{portail15}. We consider a grid of nine parameter combinations of bar angle
and pattern speed. Model names and parameter values are given in
Table~\ref{tbl:bar-specific}.

\begin{table}[ht]
\begin{center}
	\begin{tabular}{ c | c | c }
	         \toprule
	         Component & Parameter & Value \\\toprule
		Disk & $M_{\rm disk}$ & $4 \times 10^{10}~\msun$ \\
		& $a$ & 3~{\rm kpc}\\
		& $b$ & 0.28~{\rm kpc} \\\midrule
		Spheroid & $M_{\rm sph}$ & $5 \times 10^{9}~\msun$ \\
		& $c$ & 0.2 \\\midrule
	         Halo & $v_c$ & 185.8~\kms\\
		& $r_s$ & 30~kpc \\
		Bar & $M_{\rm bar}$ & $1.8 \times 10^{10}~\msun$ \\
		\bottomrule
		\end{tabular}
	\caption{The disk potential scale lengths ($a$, $b$) were adopted following
	\citep{bovy15-galpy} to match the exponential scale length of the disk
	\citep{bovyrix13} and local dark-matter density
	\citep[e.g.,][]{bovytremaine12}. The halo mass scale is set by specifying
	the circular velocity at the scale radius, $v_c$, and the scale velocity in
	Equation~\ref{eq:nfw} is given by $v_h^2 = v_c^2 / (\ln2 - 1/2)$. The bar
	mass is taken from recent 3D density modeling of red clump stars in the
	Galactic bulge \citep{portail15}. The other bar parameters are listed in
	Table~\ref{tbl:bar-specific} next to the corresponding model name.
	\label{tbl:potential-params-barred}}
\end{center}
\end{table}

\begin{table}[ht]
\begin{center}
	\begin{tabular}{ c | c | c }
	         \toprule
	         Name & $\alpha$ [deg] & $\Omega_p$ [${\rm km}~{\rm s}^{-1}~{\rm kpc}^{-1}$] \\\toprule
		bar1 & 20 & 40\\
		bar2 & 20 & 50\\
		bar3 & 20 & 60\\
		bar4 & 25 & 40\\
		bar5 & 25 & 50\\
		bar6 & 25 & 60\\
		bar7 & 30 & 40\\
		bar8 & 30 & 50\\
		bar9 & 30 & 60\\
		\bottomrule
		\end{tabular}
	\caption{Present-day bar angle ($\alpha$) and pattern speed ($\Omega_p$) for
	the nine parameter pairs considered in this work. These values span the
	range of recent measurements from a variety of techniques \citep{dwek95,
	wang12,wang13,wegg13}. \label{tbl:bar-specific}}
\end{center}
\end{table}

\subsubsection{Static potential}

For comparison, we also define a time-independent potential model with a purely
spherical bulge. In this model, we set the bar mass to 0 and instead add a
spheroidal component represented with a Hernquist potential \citep{hernquist90}.
Parameters for this potential model are given in
Table~\ref{tbl:potential-params-static}. Figure~\ref{fig:potentials}, bottom
left shows the circular velocity along the line connecting the Sun to the
Galactic center in this model (the Galactic $x$ axis).
Figure~\ref{fig:potentials}, bottom right shows contours of constant surface
density for a face-on (left) and edge-on (right) view of this potential model.

\begin{table}[ht]
\begin{center}
	\begin{tabular}{ c | c | c }
	         \toprule
	         Component & Parameter & Value \\\toprule
		Disk & $M_{\rm disk}$ & $6 \times 10^{10}~\msun$ \\
		& $a$ & 3~{\rm kpc}\\
		& $b$ & 0.28~{\rm kpc} \\\midrule
	         Halo & $v_c$ & 185.8~\kms\\
		& $r_s$ & 30~kpc \\\midrule
		Spheroid & $M_{\rm sph}$ & $1.2 \times 10^{10}~\msun$ \\
		& $c$ & 0.3 \\
		\bottomrule
		\end{tabular}
	\caption{Same as Table~\ref{tbl:potential-params-barred}, except: the disk mass is increased to account for removing the bar component, a spheroidal bulge component is added. \label{tbl:potential-params-static}}
\end{center}
\end{table}

\subsection{Fitting orbits to the Ophiuchus stream}\label{sec:ch5-orbitfit-nonapdx}

In each of the potentials described above, we fit orbits to the measured
kinematics of BHB stars that are high-likelihood members of the Ophiuchus stream
\citep{sesar15a, sesar16}. The details of this procedure and a definition of the
likelihood function we use are presented in Appendix~\ref{sec:ch5-orbitfit}. We use
an ensemble Markov Chain Monte Carlo (MCMC) algorithm \citep{goodman10}
implemented in \python\ (\package{emcee}) to generate samples from the posterior
distribution over the parameters in our orbit-fitting model
\citep{foremanmackey13}. The algorithm uses an ensemble of individual
``walkers'' to adapt to the geometry of the parameter-space being explored. In
all cases, we use 80 walkers (8 times the number of parameters).

To initialize these walkers, we first run an optimization routine to maximize
the likelihood: We use the Powell algorithm implemented in \package{Scipy}
\citep{powell64, scipy} to minimize the negative, log-likelihood. To generate
initial conditions for the walkers, we sample from Gaussian distributions
centered on the maximum likelihood values. For the coordinates, we set the
dispersions of these Gaussians to 1/1000 of the median uncertainties of the
stars. For the nuisance parameters, we set the dispersions to 1/1000 of their
maximum likelihood values.

For each potential, we run the MCMC walkers for a burn-in period of 512 steps
and then re-initialize the walkers from their positions at the end of this run.
This erases any relics of the initialization procedure outlined above. After
burn-in, we run the walkers for an additional 512 steps. For each parameter, we
compute the autocorrelation times, $\tau$, of the Markov chains and thin the
chains by taking every $2\tau$ sample. This reduces the number of samples, but
ensures that our posterior samples are effectively independent.

\subsection{Generating mock streams}\label{sec:ch5-mocks}

To generate mock stellar streams, we use the method presented in
\citet{fardal14}: Star particles are `released' from a progenitor system near
the Lagrange points with a dispersion in position and velocity that is set by
the mass and orbit of the progenitor. We draw samples from the posterior
probability distributions over orbital parameters from fitting orbits to the
stream star members and use these as the progenitor orbital parameters
(Section~\ref{sec:ch5-orbitfit}). For a given progenitor orbit---the 6D position of
the orbit today---we integrate the orbit backwards in time for a given
integration period. From the endpoint of the backwards-integration (e.g., the
past position), we begin integrating the orbit forward in time, but now at each
time-step a star particle is released near each of the Lagrange points of the
progenitor. The position of the Lagrange points and the scale of the dispersion
in position and velocity are set by the progenitor mass, $m$. The star particles
are drawn from Gaussians centered on the Lagrange points (in position) and the
progenitor (in velocity) and the full parametrization of the release
distribution is given in \cite{fardal14}. This method has been shown to
reproduce the morphologies of $N$-body simulations of stellar streams, but
requires far less computing time because it relies only on integrating
test-particle orbits.

% ---------------------------------------------------------------------------------
\begin{figure}[!tbp]
\begin{center}
%\includegraphics[width=0.5\textwidth]{figures/orbitfits}
\includegraphics[width=0.75\textwidth]{figures/ch4/orbitfits}
\caption{ Results from fitting orbits to BHB stars associated with the Ophiuchus
stream in the static and barred potential models. Data used for computing the
likelihood are shown as black points with grey error bars. The four ``fanned''
BHB stars from S16 excluded from the likelihood computation are shown as grey
squares. Error bars may sometimes be smaller than the point size. Lines (blue)
show sections of orbits integrated forward and backwards from initial conditions
drawn from the posterior samples generated by MCMC (See
Section~\ref{sec:ch5-orbitfit}). Note that the four higher dispersion stars (the
four stars with highest longitude) were not used when computing the likelihood
and are only shown for completeness. Though these four stars are significant
outliers relative to the extrapolated orbit, they are (1) at the correct
distance and sky position relative to the stream and (2) have velocities
$\approx$2.5$\sigma$ discrepant with the halo velocity distribution in this
region.}
\label{fig:orbitfits}
\end{center}
\end{figure}

We make one modification to this method based on the idea that the Ophiuchus
stream progenitor has been fully-disrupted. We add an additional parameter to
the stream generation routine to specify the time of disruption, $\tau_d$. At
this time, we set the offset of the Lagrange point to 0: In terms of the
parametrization in \citep{fardal14}, we set $k_r = 0$ and $k_{vt}=0$ but
preserve the dispersion in the release radius and velocity. Any star released
after $\tau_d$ is released with a small dispersion around the progenitor orbit
but no offset. This mass-loss history is intended to mimic the expected gradual
evaporation of a globular cluster over a tidally-limited boundary (i.e. driven
by two-body relaxation and gravitational shocks over many Gigayears) with final
disruption likely to occur once the tidal boundary is less than the core radius
of the cluster. The physics of this disruption is not followed exactly but
rather the disruption rate and final disruption time are set by the hypothesis
that the most recent combined pericenter and disk shock fully disrupted the
cluster, but, critically, that the cluster has been losing debris over its
entire orbital history.

\section{Results 1: Orbit fits and chaos}\label{sec:ch5-results1}

Figure~\ref{fig:orbitfits} summarizes our results from fitting orbits to the BHB
stream stars. Shown in each panel are the high-probability Ophiuchus stream
stars (black points, to which orbits are fit) and orbits integrated from samples
from the posterior probability over orbital parameters (blue lines). The four
``fanned'' BHB stars from S16 are shown as grey squares and are not included in
the orbit fitting procedure. We only show one of the barred potentials: The
end-to-end integration time of the orbit over the observed extent of the stream
is only $\approx$6 Myr, so the derived orbits are extremely similar in all
potentials (the time-dependence of the bar potential is not significant over
such short timescales). The orbital periods are typically $\approx$170--200 Myr
with pericenters $r_p \approx 4~{\rm kpc}$ and apocenters $r_a \approx
12$--$15~{\rm kpc}$. Though the coordinate and velocity parameter values in
observed coordinates are very similar between each potential model, the
resulting orbits are quite different. For the posterior samples in each
potential, we take the mean values of the coordinate and velocity parameters and
convert to Galactocentric coordinates (e.g., Table~\ref{tbl:param-means}).
Figure~\ref{fig:orbits-yz} shows projections of these ``mean'' orbits in each
potential model.

For the posterior samples in each potential, we also compute the maximum
Lyapunov exponent (MLE, $\lyapexp$) and corresponding Lyapunov \emph{time}
($\lyapt = 1/\lyapexp$) to assess whether each orbit is chaotic. For strongly
chaotic orbits, the Lyapunov time is still an appropriate indicator of chaos and
of the timescale over which chaos is important for tidal debris
\citep{apw15-chaos}. Figure~\ref{fig:lyapunov-hist} shows distributions of
Lyapunov times for orbits drawn from the posterior distributions from orbit
fitting in each potential model. All orbits in the static potential have
Lyapunov times $\lyapt > 20~{\rm Gyr}$ and we consider them to be regular (no
panel is shown for these orbits). All orbits sampled from each barred potential
are strongly chaotic with Lyapunov times that range from $\lyapt \approx
400$--$1100~{\rm Myr}$. It is clear from these panels that the orbits around
Ophiuchus are generally more strongly chaotic (have lower Lyapunov times) for
larger pattern speeds.

% ---------------------------------------------------------------------------------
\begin{figure}[!th]
\begin{center}
%\includegraphics[width=0.5\textwidth]{figures/orbit-yz}
\includegraphics[width=\textwidth]{figures/ch4/orbit-yz}
\caption{  Projections of orbits integrated from the mean orbital parameters
estimated from the orbit fitting posterior distributions in each potential
model. Orbits are integrated for 6 Gyr and shown in Galactocentric, Cartesian
coordinates. Even though the mean orbital parameters have nearly identical
values (e.g., the initial conditions are nearly identical in heliocentric
coordinates), the orbits in each potential model are  different in appearance. }
\label{fig:orbits-yz}
\end{center}
\end{figure}

The orbits sampled from the orbit-fit posteriors in the barred potentials are
all strongly chaotic. We have tried computing the frequency diffusion rate for
these orbits as an independent check of their chaotic timescale but have found
that, over consecutive integration windows, the frequency recovery fails or is
unreliable because the frequency spectrum changes dramatically over timescales
of $\approx$10--20 orbital periods.

% ---------------------------------------------------------------------------------
\begin{figure}[!th]
\begin{center}
%\includegraphics[width=0.5\textwidth]{figures/lyapunov-hist}
\includegraphics[width=\textwidth]{figures/ch4/lyapunov-hist}
\caption{ Histograms of estimated Lyapunov time, $\lyapt=1/\lyapexp$, for
posterior samples from the orbit fitting procedure (Section~\ref{sec:ch5-orbitfit}).
More chaotic orbits, i.e. those with shorter Lyapunov times, typically occur in
barred potentials with higher pattern speeds. There is little dependence on bar
angle. All orbits in these barred potential models are strongly chaotic with
$t_\lambda \ll t_{\rm Hubble}$ and $t_\lambda \sim t_{\rm orbit}$, where the
typical orbital period for Ophiuchus stream stars is a few hundred Megayears.}
\label{fig:lyapunov-hist}
\end{center}
\end{figure}

\section{Results 2: Stream models for the Ophiuchus stream}\label{sec:ch5-results2}

Here we study whether the observed abrupt drop in density and possible fanned
debris stars can be explained without assuming a sudden change in the mass-loss
history of the cluster. In particular, we are interested in whether the mock
streams formed around the strongly chaotic progenitor orbits in the barred
potentials can explain these features while having been steadily disrupted over
many Gigayears.

For each potential model, we randomly sample 256 orbits from the orbital
parameter posterior distributions and generate mock streams along each orbit. We
use the method outlined above (Section \ref{sec:ch5-mocks}) to generate the streams
and set the free parameters as follows: (1) we evolve the progenitors for 6 Gyr
along each orbit prior to the current position to explore stream models where
the shortness of the stream is \emph{not} due to an instantaneous disruption 400
Myrs ago, (2) we release star particles every 0.5 Myr (uniformly in time) to
densely sample the final density distribution, (3) we set the progenitor mass to
$m=10^4~\msun$ (as was estimated by S15), and (4) we set the disruption time of
the progenitor equal to the last time at which a pericenter and disk crossing
coincide (in each case, this is at $t \approx -200~{\rm Myr}$). After the
disruption time, we continue releasing the star particles uniformly in time
(every 0.5 Myr) rather than releasing a ``burst'' of particles at once. We
therefore expect that the density of the most recently disrupted debris will be
systematically higher for the model streams as compared to the observed stream.

\begin{table*}[ht]
\footnotesize
\begin{center}
	\begin{tabular}{cccccc}
	\toprule
	name & $\phi_2$ [deg] & $d$ [kpc] & $\mu_l$ [mas yr$^{-1}$] & $\mu_b$ [mas yr$^{-1}$] & $v_r$ [km s$^{-1}$]\\\midrule
	static & $-0.03\pm0.05$ & $8.3\pm0.05$ & $-7.4\pm0.1$ & $0.9\pm0.1$ & $288.9\pm0.9$\\
	bar1--9 & $-0.03\pm0.05$ & $8.35\pm0.05$ & $-7.4\pm0.1$ & $0.9\pm0.1$ & $289.0\pm1.0$\\
	\bottomrule
	\end{tabular}

	\begin{tabular}{ccc}
	\toprule
	$s_{\phi_2}$ [deg] & $s_{d}$ [kpc] & $s_{v_r}$ [km s$^{-1}$]\\\midrule
	$0.20\pm0.04$ & $0.31\pm0.10$ & $2.9\pm0.8$\\
	$0.21\pm0.05$ & $0.31\pm0.14$ & $3.2\pm0.9$\\
	\bottomrule
	\end{tabular}
	\caption{Estimated mean and standard deviation of samples from the marginal
	posterior distributions over each parameter in our orbit fit model (the
	posterior distributions are very close to Gaussian). For the barred
	potentials, all mean values are the same because the time-dependence of the
	bar doesn't impact the orbit fit over the short length of the stream. We
	have made samples from the full posterior distribution available with this
	\article\ and provide code to transform to and from stream coordinates (see
	http://adrian.pw/ophiuchus for more information).\label{tbl:param-means} }
\end{center}
\end{table*}

For each generated mock stream, we compute the likelihood of the data (now
including all BHB stars from S15 and S16, e.g., all points in
Figure~\ref{fig:orbitfits}) given the star particles by estimating the
phase-space model density using a kernel density estimate with a Gaussian kernel
\citep[see, e.g.,][]{bonaca14}. For each $i$ data point $\bs{x}_i$ and each $k$
model point with coordinates $\bs{y}_k$, we compute the likelihood by converting
the model point position and velocity into heliocentric coordinates and evaluate
\begin{align}
	p(\bs{x}_i \given \bs{y}_k, \bs{\sigma}_i, \bs{h}) = \mathcal{N}(\bs{x}_i \given \bs{y}_k, \bs{\sigma}_i^2 +\bs{h}^2)
\end{align}
where $\mathcal{N}(x\given \mu,\sigma^2)$ is the normal distribution with mean
$\mu$ and variance $\sigma^2$ and $\bs{h}$ represents the diagonal of the
bandwidth matrix, $\bs{\rm H}$, used for the density estimate, $\bs{h} = {\rm
diag}(\bs{\rm H})$. We fix the bandwidth parameters as follows
\begin{equation}
	\bs{h} = \left(
	\begin{array}{c}
	0.2~{\rm deg}\\
	0.2~{\rm deg}\\
	0.2~{\rm kpc}\\
	0.2~\kms
	\end{array}
	\right)
\end{equation} %(max$\mathcal{L}$)
for sky position in Galactic coordinates ($l$, $b$), distance, and radial
velocity. The full likelihood for all $N$ data points given $K$ model points is
\begin{equation}
	\mathcal{L} = \prod_i^N \frac{1}{K} \sum_k^K p(\bs{x}_i \given \bs{y}_k, \bs{\sigma}_i, \bs{h}).
\end{equation}

Figures~\ref{fig:mockstream0}--\ref{fig:mockstream1} show the final particle
positions and line-of-sight velocities in heliocentric coordinates for the
maximum likelihood mock streams (grey points) in each potential. Vertical,
dashed lines show the approximate extent of the part of the stream visible in
main-sequence stars (excluding the BHB stars from S16). There are a few
interesting features to note from these panels:
\begin{enumerate}
	\item Even in the static potential (Figure~\ref{fig:mockstream0}, leftmost
	column), there is a slight decrease in the density for the model stream
	towards higher Galactic longitudes, $l$. This is a projection effect: The
	portion of the stream at larger $l$ is closer and points almost directly
	towards the Sun so that the debris covers a larger area on the sky.
	\item The density of the mock stream in the static potential decreases
	slowly rather than abruptly as is observed. This is shown in the top panels
	of each column where each contour level represents a factor of 10 difference
	in projected surface density. In the static potential model the length of
	dense debris extends much farther than the observed extent of the stream
	(vertical dashed lines), whereas in some of the barred potential models the
	stars released earlier have ``fanned'' and are associated with much lower
	density debris (e.g., bar8).
	\item The four high-dispersion BHB stars beyond the end of the stream (from
	S16) don't match in position and velocity with the particle distribution
	from even the maximum likelihood stream model in the static potential. In
	some barred potentials the chaotic evolution of the stream stars can lead to
	over-densities of stars with an increased positional dispersion and
	significantly discrepant velocities (e.g., bar8).
	\item None of the stream models---static or barred---produce an appreciable
	density of stars with line-of-sight velocities near the S16 BHB star with
	the largest velocity ($\approx 320~\kms$). This star is either an
	interesting Ophiuchus member star or is associated with some other kinematic
	substructure.
	\item For the barred potentials, the stream morphology is very sensitive to
	the properties of the bar (especially the pattern speed) and to the initial
	conditions of the orbit. We have found that the morphology can vary
	significantly between nearby orbits in the same potential model (because
	these are strongly chaotic orbits), but the overall characteristics remain
	similar: along more strongly chaotic orbits, the debris ``fans'' more and
	the apparent dense part of the model streams is shorter.
\end{enumerate}

The density truncation of the mock streams in each potential model is more
clearly seen in terms of the density contrast between stream stars and
background stars, visualized in Figure~\ref{fig:densitymaps}: This figure shows
mock sky-density maps of stars generated by superimposing the maximum likelihood
model streams over a noisy background of stars. The number of mock stream star
particles used to generate the map has been normalized such that the total
number of stars within the observed extent of the stream (between $5.85 > l >
3.81~{\rm deg}$) is equal to the number of stars attributed to the Ophiuchus
stream in the PS1 data \citep[$N \approx 500$][]{bernard14}.

The viewing angle and stream geometry is more clearly demonstrated in
Figure~\ref{fig:mockstreamxyz}, which shows $x$-$z$ projections of the star
particles in Galactocentric Cartesian coordinates (grey) along with the position
of the Sun (symbol at $(x,z)=(-8.3,0)~{\rm kpc}$) and the ``window'' of the
heliocentric, sky-position plots of
Figures~\ref{fig:mockstream0}--\ref{fig:mockstream1} (shown as blue lines).

\section{Discussion}\label{sec:ch5-discussion}

The model streams presented here do not reproduce all observed features of the
Ophiuchus stream (the details of the potential and the orbit predict vastly
different phase-space morphologies for the fanned part). Instead, these results
illustrate that chaotic evolution of tidal debris can plausibly explain the
peculiar features of the stream. If the cluster progenitor was on a regular
orbit, it would have to have disrupted entirely within the last
$\approx$300--600 Myr in order to explain the shortness and density profile. In
addition, the four most recently identified BHB stars with similar distances and
line-of-sight velocities would have to be (highly unlikely) chance alignments of
halo stars. If instead the progenitor were on a chaotic orbit (because of the
influence of the Galactic bar): (1) stars stripped early will have ``fanned''
out and would thus be harder to observe and (2) the nearby,
high-velocity-dispersion BHB stars can be naturally explained by this chaotic
stream-fanning. We consider the second scenario to be more plausible: our
understanding of the formation and evolution of the Ophiuchus stream is that the
progenitor object has been orbiting and steadily losing stars over the last
several Gigayears, but only the stars stripped from the most recent few
pericentric passages remain coherent enough to be detected as a stream-like
over-density in the PS1 data.

It is still too early to say for sure that chaotic stream-fanning is occurring
for the Ophiuchus stream. Deeper follow-up imaging and spectroscopy over a
larger area region around the stream will be needed to test the predictions of
this work and compare with other possible scenarios. For example, recent work
has shown that if the Ophichus stream progenitor orbit is in resonance with the
bar, the debris can remain short for at least 1 Gyr \citep{hattori15}. In their
model, there would be no nearby, high-dispersion debris, and the pattern speed
of the bar would be related to the orbital frequencies of the progenitor orbit.
However, it has not been demonstrated whether this proposed scheme can explain
the shortness of the stream over timescales closer to the age of the stellar
population ($\approx$10 Gyr). With more information about the density
distribution of stars in the stream and better proper motion measurements we
would be able to (1) help distinguish models for the Milky Way bar independent
from current methods and (2) begin to model the survivability of globular
clusters orbiting in the central Milky Way \citep[e.g.,][]{gnedin97}.

\subsection{Future work: Modeling the Galactic bar}

The current kinematic data for the Ophiuchus BHB stars suggests that the stream
is sensitive to the gravitational potential of the Milky Way bar---with better
measurements of the velocities and a larger sample of member stars we will
constrain parameters for the bar model. Current measurements of the pattern
speed, angle, and structure of the Milky Way's bulge and bar are largely
discrepant \citep[e.g.,][]{wang12, wang13, wegg13, antoja14}. Most of the
methods that infer these quantities rely on modeling the density or kinematics
of stars at low Galactic latitudes and must therefore handle challenges with
completeness and dust extinction. The Ophiuchus stream offers a unique
opportunity to independently measure these quantities by modeling the density
and kinematics of stars associated with the stream.

\subsection{Future work: The orbits and survivability of inner Milky Way globular clusters}

If the Ophiuchus stream formed from a globular cluster on a chaotic orbit and we
happen to be witnessing its final demise, what does this imply about the
population of clusters that have already been fully disrupted? The existence of
strongly chaotic orbits in this region would limit the expected number of cold
stellar streams in the inner Galaxy and enhance the rate of mixing of the
debris. The fraction of strongly chaotic orbits that would lead to chaotic
fanning or fast dispersal of tidal debris should therefore be related to the
amount of kinematic substructure in the inner Galaxy. Indeed, first suggestions
of kinematic substructure in the bulge have been found, but further modeling is
needed to understand whether these hints could be signature of a widely
dispersed globular cluster population. If so, a stronger theoretical
understanding of the prevalence of these features could be combined with future
kinematic surveys (from, e.g., \project{Gaia}) to place constraints on
long-standing puzzles about the primordial globular cluster population
\citep[e.g.,][]{murali97, gnedin97}.

\section{Conclusions}\label{sec:ch5-conclusions}

We have shown that, with a qualitative but observationally-motivated potential
model for the Galactic bar, the orbits of the Ophiuchus stream stars are likely
sensitive to the time-dependence and shape of the bar potential. For modeling
the stream density itself, it is therefore crucial to include this component of
the Galactic potential. By fitting orbits to kinematic data for members of the
stream in Milky Way-like potential models, we have found that orbits in the
vicinity of the Ophiuchus stream are strongly chaotic for a range of bar
parameters (pattern speeds and present-day angles). Using mock stellar stream
models generated assuming a globular cluster-mass progenitor object, we have
shown that the apparent shortness of the stream and the existence of nearby
stars with very high velocity dispersion are plausibly explained by chaotic
density evolution of the stars stripped from the progenitor object.

This is the first time chaos has been used to explain the morphology of a
stellar stream and the first observational evidence for the importance of chaos
in the Galactic halo. It also highlights the importance of including the
Galactic bar in dynamical modeling of the Milky Way's inner halo and has
important implications for future modeling of streams near in this region. With
more Ophiuchus stream members, density and velocity information over a larger
region near the stream, and better models for the internal structure of the
Galactic bar, careful modeling of this stream could lead to tight constraints on
the structure and evolution of the bar.

\section*{Acknowledgements}
APW is supported by a National Science Foundation Graduate Research Fellowship
under Grant No.\ 11-44155. KVJ and APW acknowledge support from NSF grant
AST-1312196. This material is based on work supported by the National
Aeronautics and Space Administration under Grant No.\, NNX15AK78G issued through
the Astrophysics Theory Program. APW acknowledges the staff at the MPIA for
their support and assistance. The authors wish to acknowledge Victor Debattista,
Melissa Ness, and Sarah Pearson for useful discussions. APW acknowledges David
Bowie (1947--2016) for his continuous inspiration and artistic vision. This
research made use of Astropy, a community-developed core Python package for
Astronomy \citep{astropy13}. This work additionally relied on Columbia
University's \emph{Yeti} compute cluster, and we acknowledge the Columbia HPC
support staff for assistance. \\

% ---------------------------------------------------------------------------------


\begin{figure*}[p]
\begin{center}
\includegraphics[width=\textwidth]{figures/ch4/mockstream0}
\caption{ Sky position, distance, and line-of-sight velocity in heliocentric,
Galactic coordinates for star particles (contours or grey points) from the
maximum-likelihood mock streams in each potential. Top panels show surface
density of mock stream star particles in each potential. Contours are spaced
logarithmically from $10^{-2}$ to $10$ particles per sq. deg---that is, each
color represents a factor of 10 difference in surface density. In the static
potential (top left), the density remains high along the center of the stream,
but for some of the barred potentials the density drops sharply because of
chaotic stream-fanning. Vertical, dashed lines show the approximate extent of
the densest part of the stream visible in main-sequence stars \citep[the segment
originally detected in ][]{bernard14}.}
\label{fig:mockstream0}
\end{center}
\end{figure*}


\begin{figure*}[p]
\begin{center}
\includegraphics[width=\textwidth]{figures/ch4/mockstream1}
\caption{ Same as Figure~\ref{fig:mockstream0} for the other five barred potentials. }
\label{fig:mockstream1}
\end{center}
\end{figure*}

% ---------------------------------------------------------------------------------

\begin{figure*}[p]
\begin{center}
\includegraphics[width=\textwidth]{figures/ch4/densitymaps}
\caption{ Simulated maps of the 2D density of star particles from the
maximum-likelihood mock streams in each potential with a noisy background of
stars binned into 10' by 10' pixels. The background star density is assumed to
be Poisson with $\lambda = 42$ \citep[see Figure 3 in][where the typical
background density is $\approx\frac{60}{(0.2~{\deg})^2}$]{bernard14}. The mock
stream particles are down-sampled so that the total number of particles in the
region of sky that the stream is seen as an over-density matches the observed
number of stars \citep[$N\approx500$][]{bernard14}. Color-scale is stretched so
that white to black is 2nd to 98th percentile.}
\label{fig:densitymaps}
\end{center}
\end{figure*}

% ---------------------------------------------------------------------------------

\begin{figure*}[p]
\begin{center}
\includegraphics[width=\textwidth]{figures/ch4/mockstream-xyz}
\caption{ Star particles (grey points) from mock streams generated on the mean
orbits in each potential model shown in projections of Galatocentric, Cartesian
coordinates. The position of the Sun is shown as the symbol at
$(x,z)=(-8,0)~{\rm kpc}$. The volume of the sky position and distance plots of
Figures~\ref{fig:mockstream0}--\ref{fig:mockstream1} are shown transformed to
these coordinates as the blue wedge near $(x,z)=(-2,4)~{\rm kpc}$. This
demonstrates that the stream is nearly aligned our viewing angle. }
\label{fig:mockstreamxyz}
\end{center}
\end{figure*}

% ---------------------------------------------------------------------------------

% TODO: appendix
\setcounter{section}{0}%
\renewcommand\thesection{\thechapter.\Alph{section}}

\section{Transformation from Galactic to Ophiuchus stream coordinates} \label{sec:ch5-rotationmatrix}
The transformation matrix is approximately represented as
\begin{equation*}
\left( \begin{array}{c}
x \\
y \\
z \end{array} \right)_{\rm Oph} \approx
\left( \begin{array}{ccc}
0.84922096554 & 0.07001279040 & 0.52337554476\\
-0.27043653641 & -0.79364259852 &  0.54497294023\\
0.45352820359 & -0.60434231606 & -0.65504391727
\end{array} \right) \,
\left( \begin{array}{c}
x \\
y \\
z \end{array} \right)_{\rm Gal}
\end{equation*}
but the precise transformation and coordinate frame is implemented in \python\ using the \package{Astropy} coordinates package.\footnote{See \url{http://adrian.pw/ophiuchus} for more information}  This code is hosted on \project{GitHub}.\footnote{\url{https://github.com/adrn/ophiuchus}}

\section{Fitting orbits to stellar streams}\label{sec:ch5-orbitfit}

Our goal is to infer the posterior probability distributions over orbital
initial conditions, $\bs{w}_0=(l, b, \DM, \mu_l, \mu_b, v_r)_0$, given a
potential, $\Phi$, and kinematic data for each $i$ stream star, $\bs{x}_i=(l, b,
\DM, \mu_l, \mu_b, v_r)_i$. In this notation, $(l, b)$ are Galactic coordinates,
$\DM$ is the distance modulus, $(\mu_l, \mu_b)$ are proper motions in the
Galactic frame, and $v_r$ is the radial velocity. We assume that the sky
coordinates for each star are known perfectly well (have zero uncertainty) and
transform the data to a rotated, heliocentric coordinate system that is aligned
with the stream and centered on the median sky position of the BHB stars in the
densest part of the stream \cite[all BHB stars except the `fanned' stars:
cand15, cand26, cand49, cand54 from][]{sesar16}. We represent the longitude and
latitude in these coordinates as $(\phi_1, \phi_2)$ and the rotation matrix to
transform from Galactic to these coordinates is given in
Appendix~\ref{sec:ch5-rotationmatrix}. We treat the stream longitude, $\phi_1$, as
the perfectly-known, independent variable so that all other coordinates can be
expressed as functions of this longitude (e.g., $\phi_2(\phi_1)$, ${\rm
DM}(\phi_1)$, etc.). This methodology is similar to that used in
\cite{koposov10} and \cite{sesar15a}.

\subsection{Likelihood}

We include three nuisance parameters in our likelihood to account for the
internal dispersion of the stream: in observed coordinates, these are the on-sky
positional dispersion, $s_{\phi_2}$, a distance (modulus) dispersion, $s_{\DM}$,
and a radial velocity dispersion, $s_{v_r}$ (the proper motion uncertainties are
sufficiently large that we can't resolve the velocity dispersion in these
coordinates).\footnote{We assume that the dispersion in these coordinates is
constant over the observed (short) section of the stream. This may be a bad
assumption.} We add two additional nuisance parameters for controlling the
amount of time to integrate forwards, $t_f$, and backwards, $t_b$, from the
given initial conditions, which ultimately controls the length of the section of
orbit that is compared to the stream star data. For brevity in the equations
below, we define $\bs{s} = (s_{\phi_2}, s_{\DM}, s_{v_r})$ and $\bs{\theta} =
(\bs{w}_0, \Phi, t_b, t_f)$.

For a given set of initial conditions ($\bs{w}_0$), we compute a model orbit as
follows: (1) transform the initial conditions to Galactocentric coordinates, (2)
integrate the orbit forward and backward by $t_f$ and $t_b$, respectively, in
the potential $\Phi$, (3) transform all orbit points (time-steps) back to
observed coordinates, and (4) define interpolating functions for each coordinate
as a function of stream longitude, $\phi_1$, using cubic splines---e.g.,
functions $\widetilde{\phi}_{2}(\phi_1)$, $\widetilde{\DM}(\phi_1)$,
$\widetilde{\mu_l}(\phi_1)$, $\widetilde{\mu_b}(\phi_1)$,
$\widetilde{v_r}(\phi_1)$. These functions let us compute the predicted values
of each of these coordinates at the longitudes of each observed star,
$\phi_{1,i}$.

We assume that each observed kinematic component is independent so that the
likelihood of the data for a given star, $\bs{x}_i$, with uncertainties,
$\bs{\sigma}_i$, is given by the product over the likelihoods for each dimension
of the data:
\begin{multline}
	p(\bs{x}_i \given \bs{\sigma}_i, \bs{s}, \bs{\theta}) = p(\phi_{2,i} \given \phi_{1,i}, s_{\phi_2},\bs{\theta}) \, p(\DM_i \given \phi_{1,i}, \sigma_{\DM,i}, s_\DM, \bs{\theta})\\
	\times p(\mu_{l,i} \given \phi_{1,i}, \sigma_{\mu_{l},i}, \bs{\theta}) \, p(\mu_{b,i} \given \phi_{1,i}, \sigma_{\mu_{b},i}, \bs{\theta}) \, p(v_{r,i} \given \phi_{1,i}, \sigma_{v_r,i}, s_{v_r}, \bs{\theta}).
\end{multline}
The uncertainties in these observed coordinate components are assumed to be
normally distributed away from the model values: using the notation
\begin{align}
	\norm(x \given \mu, \sigma^2) &= \frac{1}{\sqrt{2\pi \sigma^2}} \, \exp\left(-\frac{(x-\mu)^2}{2\sigma^2}\right)
\end{align}
the likelihoods are
\begin{align}
	p(\phi_{2,i} \given \phi_{1,i}, s_{\phi_2},\bs{\theta}) &= \norm(\phi_{2,i} \given \widetilde{\phi}_{2}(\phi_{1,i}), s^2_{\phi_2})\\
	p(\DM_i \given \phi_{1,i}, \sigma_{\DM,i}, s_\DM, \bs{\theta}) &= \norm(\DM_i \given \widetilde{\DM}(\phi_{1,i}), s^2_{\DM} + \sigma^2_{\DM,i})\\
	p(\mu_{l,i} \given \phi_{1,i}, \sigma_{\mu_{l},i}, \bs{\theta}) &= \norm(\mu_{l,i} \given \widetilde{\mu_{l}}(\phi_{1,i}), \sigma^2_{\mu_{l,i}})\\
	p(\mu_{b,i} \given \phi_{1,i}, \sigma_{\mu_{b},i}, \bs{\theta}) &= \norm(\mu_{b,i} \given \widetilde{\mu_{b}}(\phi_{1,i}), \sigma^2_{\mu_{b,i}})\\
	p(v_{r,i} \given \phi_{1,i}, \sigma_{v_r,i}, s_{v_r}, \bs{\theta}) &= \norm(v_{r,i} \given \widetilde{v_r}(\phi_{1,i}), s^2_{v_r} + \sigma^2_{v_r,i}).
\end{align}
We assume the data from each star is independent and identically distributed
(i.i.d.) so that the full likelihood is the product over the likelihoods for all
$N$ stars:
\begin{equation}
	 p(\{\bs{x}_i\} \given \{\bs{\sigma}_i\}, \bs{s}, \bs{\theta}) = \prod_i^N p(\bs{x}_i \given \bs{\sigma}_i, \bs{s}, \bs{\theta}).\label{eq:likelihood}
\end{equation}

\subsection{Priors}

For the intrinsic dispersion parameters, we use logarithmic (scale-invariant)
priors such that $p(s) \propto s^{-1}$. For the integration time parameters, we
use uniform priors, $\mathcal{U}(a,b)$ (over the range $a$--$b$),
\begin{align}
	p(t_f) &= \mathcal{U}(1,100)~{\rm Myr}\label{eq:prior1}\\
	p(t_b) &= \mathcal{U}(-100,-1)~{\rm Myr}.
\end{align}
Note that present-day is $t=0$. For computational efficiency, we place strong
priors on the minimum and maximum longitudes of the model points, $(\phi_{1,{\rm
min}},\phi_{1,{\rm max}})$ so that the model orbit does not integrate for longer
than necessary. In particular, we set
\begin{align}
	p(\phi_{1,{\rm min}} \given \bs{\theta}) &= \norm(\phi_{1,{\rm min}} \given \min(\phi_{1,i}), s^2_{\phi_2})\\
	p(\phi_{1,{\rm max}} \given \bs{\theta}) &= \norm(\phi_{1,{\rm max}} \given \max(\phi_{1,i}), s^2_{\phi_2}).
\end{align}
For the orbital initial condition components, we use uniform priors in each
cartesian position component over the range $(-200,200)~{\rm kpc}$. For
velocity, we use a Gaussian prior on the magnitude of the total velocity, $v$,
with a dispersion of $150~\kms$,
\begin{equation}
	\norm(v \given 0, (150~\kms)^2) \label{eq:prior2}
\end{equation}
We keep the potential, $\Phi$, fixed. In total, this model has 10 parameters (5
phase-space coordinates, 5 nuisance parameters).

The full expression for the posterior probability, $p(\bs{s}, \bs{w}_0, t_b, t_f
\given \{\bs{x}_i\}, \{\bs{\sigma}_i\}, \Phi)$, is the joint likelihood
(Equation~\ref{eq:likelihood}) multiplied by all priors described above
(Equations~\ref{eq:prior1}--\ref{eq:prior2}).

\renewcommand\thesection{\thechapter.\arabic{section}}
