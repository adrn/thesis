\thispagestyle{empty}
\begin{center}

{\Large \bf ABSTRACT}

\vspace{.35in}
{\large \bf \thesistitle}

\vspace{.35in}

{\large Adrian M. Price-Whelan} \\
\vspace{.35in}
\end{center}

We develop two new methods to measure the structure of matter around the Milky
Way using stellar tidal streams from disrupting dwarf galaxies and globular
clusters. The dark matter halo of the Milky Way is expected to be triaxial and
filled with substructure, but measurements of the shape and profile of dark
matter around the Galaxy are highly uncertain and often contradictory. We
demonstrate that kinematic data from near-future surveys for stellar streams or
shells produced by tidal disruption of stellar systems around the Milky Way will
provide precise measures of the gravitational potential to test these
predictions. We develop a probabilistic method for inferring the Galactic
potential with tidal streams based on the idea that the stream stars were once
close in phase space and test this method on synthetic datasets generated from
N-body simulations of satellite disruption with observational uncertainties
chosen to mimic current and near-future surveys of various stars. We find that
with just four well-measured stream stars, we can infer properties of a triaxial
potential with precisions of order 5--7 percent.

We then demonstrate that, if the Milky Way's dark matter halo is triaxial and is
not fully integrable (as is expected), an appreciable fraction of orbits will be
chaotic. We examine the influence of chaos on the phase-space morphology of cold
tidal streams and show that streams even in weakly chaotic regions look very
different from those in regular regions. We discuss the implications of this
fact given that we see several long, thin streams in the Galactic halo; our
results suggest that long, cold streams around our Galaxy must exist only on
regular (or very nearly regular) orbits and potentially provide a map of the
regular regions of the Milky Way potential. We then apply this understanding of
stream formation along chaotic orbits to the interpretation of a
newly-discovered, puzzling stellar stream near the Galactic bulge. We conclude
that the morphology of this stream is consistent with forming along chaotic
orbits due to the presence of the time-dependent Galactic bar.

These results are encouraging for the eventual goal of using flexible,
time-dependent potential models combined with larger data sets to unravel the
detailed shape of the dark matter distribution around the Milky Way.
