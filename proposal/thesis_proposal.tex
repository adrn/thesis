\documentclass[letterpaper,12pt,preprint]{aastex}

% packages
\usepackage{amssymb,amsmath}

\begin{document}

\title{Measuring the Gravitational Potential of the Milky Way with Stellar Debris}
\author{Adrian M. Price-Whelan}

\section{Introduction}

A great surprise from the last century of concerted efforts in both cosmological theory and observation is that familiar, baryonic matter is, perhaps, the exception: [...] by cosmic acceleration due to \emph{dark energy} and dwarfed in mass by the apparent existence of \emph{dark matter}, it seems that ordinary atoms and interactions are inconsequential in the history of the Universe on the largest scales. [...sentence about dark energy...] On galactic scales, the effect of cosmic expansion is masked by the gravitational influence of a network of vast, yet unseen dark matter \emph{halos} that enshroud and support the galaxies that they contain. Cosmological, dark-matter-only simulations of structure formation have demonstrated the dominance of the dark sector on large ($\gtrsim$100~kpc) scales by, e.g., reproducing the observed clustering of galaxies with impressive accuracy (cite many). These simulations also predict universal mass profiles and shapes of dark matter halos \citep{navarro96, others}. However, simply including baryons as passive followers of the dark matter does not carry this consistency to galactic ($\lesssim$10~kpc) scales. It seems that the processes involved in galaxy formation and evolution are not simply dictated by the dark matter: relaxation and energetic feedback from the baryons has been shown to alter the interior structure of halos \citep[e.g.][]{bailin05, pontzen12} (more, maybe also a debatista paper? flattening of potential), calling to question the universality of dark matter halo properties. Hence, detailed measurements of the shape, orientation, radial profile, and extent of dark matter halos provides information about the cosmological formation of these structures, as well as the internal baryonic processes that continue to shape them.

Measurements of unseen dark matter on large scales rely on observations of a number of baryonic probes, e.g., the orbits of stars and gas in the Milky Way (rubin 1970, other papers on the rotation curve / Oort problem), large-scale interactions between galaxy clusters (bullet cluster), and gravitational lensing of galaxies by foreground mass (). The earliest work on constraining (and discovering) the dark matter density used the vertical [...] of stars in the solar neighborhood [...] \citep[now known as the \emph{Oort Problem},][]{oort32}. More recently, \cite{bovy12} have repeated [...]. Evidence for the existence of dark matter later resurfaced with the measurement of the rotation curve of M31 \citep{rubin70} and discovery that it remains flat [...]. Using the method proposed in \cite{merrifield92}, [...] finally measured the rotation curve of the Milky Way and again found [...] deviated from Keplerian. With [...], [...] used lensed galaxies to reconstruct the 2D, projected mass density of [several] foreground galaxies, providing one of the tightest and most convincing measurements for the existence of dark matter halos. 

The most precise measurements of the shape and profile of a dark matter halo comes from the Milky Way itself, where our unique perspective allows a three dimensional analysis of the mass distribution in the halo. Attempts to study the detailed distribution of mass in our own dark matter halo at large distances use stellar tracers and either (1) assume the tracers are sampled from a random, phase-mixed halo population, or (2) exploit known overdensities of stars that form coherent, dynamical substructure. 

[ Jeans modeling explain latest \cite{deason12} results ]

[ Nbody modeling to reproduce Sgr ]

(*all are important* - confirmations of each other)

[ Overview of satellites and substructure in the halo ]

[ Status of software in astronomy ]

[ Overview of outreach activities etc? ]

\section{\texttt{REWIND}}

[ Discussion of other modeling techniques ]
[ - Deason et al. tracers ]
[ - Action-angle stuff (not easily applied to *real data*) ]
[ - Law \& Majewski model, why Nbody is tough ]

[ We want to model the potential that: ]
[ - exploits the substructure and kinematically associated stars ]
[ - can run on real observational data, with missing dimensions, no assumptions about analytic form of potential ]
[ - avoids building and running a fully dynamical model of a disrupting satellite (e.g., Nbody simulation for every step in parameter space) ]

\section{SMASH RR Lyrae}

[ Steal text from proposal? ]

\section{Triangulum-Andromeda}

[ Observational side project ... ]

\section{Astropy}

[ All along the way, developing code that can be generalized and used by the community as a whole ]

\section{Outreach and Teaching}

[ Interactive web tools ]
[ Scicoder ]
[ Python workshops ]

\bibliographystyle{apj}
\bibliography{refs}

\end{document}
