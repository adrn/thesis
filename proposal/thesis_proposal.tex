\documentclass[letterpaper,12pt,preprint]{aastex}

% packages
\usepackage{amssymb,amsmath}

\begin{document}

\title{Measuring the Gravitational Potential of the Milky Way with Stellar Debris}
\author{Adrian M. Price-Whelan}

\section{Introduction}

A great surprise from the last century of concerted efforts in both cosmological theory and observation is that familiar, baryonic matter is, perhaps, the exception: dwarfed in mass by the apparent existence of \emph{dark matter} and even further diminished by cosmic acceleration due to \emph{dark energy}, it seems that ordinary atoms and interactions are inconsequential to the history of the Universe on the largest scales. [...sentence about dark energy...] On galactic scales, the effect of cosmic expansion is masked by the gravitational influence of a network of vast, yet unseen dark matter \emph{halos} that enshroud and support the galaxies that they contain. While cosmological, dark-matter-only simulations of structure formation have demonstrated  the dominance of the dark sector on large ($\gtrsim$100~kpc) scales (cite many), simply including baryons as passive followers of the dark matter does not carry this consistency to galactic ($\lesssim$10~kpc) scales. It now seems that the processes involved in galaxy formation and evolution are not simply dictated by the dark matter: relaxation and energetic feedback from the baryons has been shown to alter the interior structure of halos \citep[e.g.][]{bailin05, pontzen12} (more, maybe also a debatista paper? flattening of potential), calling to question the universality of dark matter halo profiles and shapes. Hence, detailed measurements of the shape, orientation, radial profile, and extent of dark matter halos provides information about the cosmological formation of these vast structures, as well as the smaller-scale baryonic processes that continue to shape them.

%However, the firmness of these conclusions about the unseen dark sector rely on observations of a number of baryonic probes, e.g., the orbits of stars and gas in the Milky Way (rubin 1970, other papers on the rotation curve / Oort problem), large-scale interactions between galaxy clusters (bullet cluster), gravitational lensing of galaxies (), the clustering of galaxies (), and pulsating (cite cepheid) and exploding (cite perlmutter/riess) stars.

% by statistically reproducing the observed clustering of galaxies
%with impressive accuracy (cite) and also revealing the ubiquity and universal properties of dark matter halos \citep{navarro96} (others?). However, 

[ What has been done? ]

[ - initial work on DM density / Oort problem ]
	Earliest work on local dark matter density by \cite{oort32}, more recently \cite{bovy12}

[ - rotation curve ]
	\cite{rubin70} for andromeda
	updated method to measure the rotation curve of the Milky Way \cite{merrifield92}
	
[ - 2D projected mapping w/ lensing ]

[ - kinematic tracer populations ]
	\cite{deason12} with BHB stars from SDSS

[ - substructure ]
	Sgr and etc.

(*all are important* - confirmations of each other)

[ Overview of satellites and substructure in the halo ]

[ Status of software in astronomy ]

[ Overview of outreach activities etc? ]

\section{\texttt{REWIND}}

[ Discussion of other modeling techniques ]
[ - Deason et al. tracers ]
[ - Action-angle stuff (not easily applied to *real data*) ]
[ - Law \& Majewski model, why Nbody is tough ]

[ We want to model the potential that: ]
[ - exploits the substructure and kinematically associated stars ]
[ - can run on real observational data, with missing dimensions, no assumptions about analytic form of potential ]
[ - avoids building and running a fully dynamical model of a disrupting satellite (e.g., Nbody simulation for every step in parameter space) ]

\section{SMASH RR Lyrae}

[ Steal text from proposal? ]

\section{Triangulum-Andromeda}

[ Observational side project ... ]

\section{Astropy}

[ All along the way, developing code that can be generalized and used by the community as a whole ]

\section{Outreach and Teaching}

[ Interactive web tools ]
[ Scicoder ]
[ Python workshops ]

\bibliographystyle{apj}
\bibliography{refs}

\end{document}
